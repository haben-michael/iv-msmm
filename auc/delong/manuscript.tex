\documentclass[12pt]{article}
\usepackage{amsthm}
\usepackage{bbm}
\usepackage{amssymb}
\usepackage{mathtools}
\mathtoolsset{showonlyrefs,showmanualtags}
\usepackage{etoolbox}
% \usepackage{booktabs}
% \usepackage{url}
%\usepackage{setspace}
\usepackage[margin=1in]{geometry}
% \usepackage{authblk}
\usepackage{natbib}
% \usepackage[page]{appendix}
\usepackage[nomarkers,nolists]{endfloat}
\usepackage{graphicx}
% \usepackage{tikz}
% \usetikzlibrary{calc}
\usepackage{subfig}
\usepackage{enumitem}
% \usepackage{array} % tables with fixed width and alignment
\DeclareMathOperator{\AUC}{AUC}
\DeclareMathOperator{\V}{Var}
\DeclareMathOperator{\cov}{Cov}
\DeclareMathOperator{\corr}{Corr}
\DeclareMathOperator{\sd}{sd}
\newcommand{\E}{E}
\renewcommand{\P}{P}
\newcommand{\cind}{\perp \!\!\! \perp}
\newcommand{\X}[1][]{X_{0#1}}
\newcommand{\Y}[1][]{X_{1#1}}
\newcommand{\W}[1][]{W_{#1}}
\newcommand{\w}[1][]{w_{#1}}
\newcommand{\z}[1][]{w_{#1}}
\newcommand{\D}[1][]{D_{#1}}
\renewcommand{\t}[1]{{#1}^T}
\renewcommand{\star}[1]{{#1}^\ast}
\newcommand{\infl}[1][]{\phi_{#1}}
\newcommand{\F}{F}
\newcommand{\G}{G}
% \newcommand{\D}{D}
\newcommand{\m}{m}
\newcommand{\n}{n}
\newcommand{\N}{m+n}
% \newcommand{\risk}{\rho}
\newcommand{\risk}[1][]{\rho_{#1}}
\newcommand{\auc}{\theta}
% \newcommand{\betastar}{\beta_0}
\newcommand{\aucdiff}{\Delta\text{AUC}}
\newcommand{\aucdiffhat}{\hat{\Delta\text{AUC}}}
\newcommand{\kernel}[2]{\{#1 < #2\}}
% \newcommand{\infl}{\phi}
\newcommand{\h}{h}
\newcommand{\termb}{term 2 }
\newtheorem{theorem}{Theorem}
\newtheorem{proposition}[theorem]{Proposition}
\newtheorem{lemma}[theorem]{Lemma}
\newtheorem{corollary}[theorem]{Corollary}
\theoremstyle{definition}
\newtheorem{example}{Example}%[section]
\makeatletter
% \def\input@path{{input/}{figs/}}
% \graphicspath{{./figs/}}
\newtoggle{commenttoggle}
\togglefalse{commenttoggle}
\newcommand{\comment}[1]{
  \iftoggle{commenttoggle}{
    {\normalsize{\color{red}{ #1}}\normalsize}
  }
  {}
}
\title{Nonparametric estimation of the auc of an index estimated from a single sample}
% \author[1]{Haben Michael}
% \author[2]{Lu Tian}
% \affil[1]{University of Massachusetts}
% \affil[2]{Stanford University}
\date{}

\begin{document}
\maketitle
((previously: ``auc of an estimated index'' but that isn't actually the target of estimation))

Abstract. We describe a nonparametric method of estimating the AUC of
an index $\t\beta x$ when $\beta$ is estimated from the same data, with a
focus on nonparametric estimation of the difference of the AUCs of two
distinct indices.

\section{Introduction}

measuring the discrimination of biomarkers using the auc. $\aucdiff$ to
measure the difference in discrimination between two markers. biomarkers--index auc. often
one index is based on a subset of covariates on which the other is based.

unexpected behavior when comparing $\aucdiff$ to tests on the indexes
themselves. maybe mention ``bafflement'' over test behavior. at the
begining of 2010s.

focus here on alternative hypothesis.

useful cites: 
demler 2017: $\aucdiff$ is one of the most widely used measures of
discrimination.((should be ``difference'' in discrimination I
think?)). maybe cite seshan counts of clinical papers, sas proc. ((
seshan 2013: In the first four months of 2011 alone, we easily
identified seven articles in clinical journals that used the AUC test
to compare nested logistic regression models [11, 12, 13, 14, 15, 16,
17]))


\section{Background/setting}


% difference of 2 aucs.
% may be viewed as a U-statistic.

Data model: Pair $(\W,\D)$, $\W\in\mathbb{R}^p$ ((continuous?)), $\D\in\{0,1\}$. Denote by $\X\sim\F,\Y\sim\G$ the RVs and distributions obtained by conditioning $\W$ on $\D=0$ and $\D=1$.

Let $((\W[1],\D[1]),\ldots,(\W[\N],\D[\N]))$ be an IID sample according to ((ref)), with the control and case variables 
\begin{align}
  \X[1],\ldots,\X[\m] \sim \F,  IID,   \Y[1],\ldots,\Y[\n] \sim \G, IID.
\end{align}

% Based on this data, analyst obtains coefficients $\beta,\gamma$.
Vectors $\hat\beta \in \mathbb{R}$ and $\hat\gamma \in \mathbb{R}$ are
obtained based on the sample (the ``coefficient estimation procedure''). They are assumed to have finite 
probability limits $\star\beta$ and $\star\gamma$ as $n\to\infty$. ((need to assume nonzero?))

The statistic $\aucdiffhat$ is
\begin{align}
  \frac{1}{\m\n}\sum_{i,j}\kernel{\t{\hat\beta}\X[i]}{\t{\hat\beta}\Y[j]}
  -  \frac{1}{\m\n}\sum_{i,j}\kernel{\t{\hat\gamma}\X[i]}{\t{\hat\gamma}\Y[j]} 
\end{align}


An explicit probabilty model may not be specified, and often the 
estimation methods for $\hat\beta$ and $\hat\gamma$ imply inconsistent models. E.g., logistic models
with nonzero covariates omitted from the reduced model ((ref logit example below))
 Nevertheless
inference is sought ((cite applied papers)), particularly 1. whether
the difference in the AUCs of the two markers $\t{\hat\beta} x$ and
$\t{\hat\gamma} x$ is in some limiting sense  nonzero, and if so
2. the magnitude of the difference. Assume here that limiting sense is the difference of AUCs of the indexes at the starred parameters, $\t{\star\beta} x$ versus $\t{\star\gamma} x$:
\[
  \aucdiff=...
\]
% For fixed $\hat\beta,\hat\gamma$, 

The statistic ((ref)) may be viewed as a process of two-sample U-statistics with
kernel
$(x,y)\mapsto \kernel{\t\beta x}{\t\beta y} - \kernel{\t\gamma x }{ \t\gamma y}$, indexed by $\beta,\gamma$, and
evaluated at the random vectors $\hat\beta,\hat\gamma$. This statistic
presents two complications for basic U-statistics theory.

1. Under the null of no diff, $\aucdiff=0$, the statistic ((ref)) is
often a degenerate U-statistic. The asy distribution of a degenerate
U-statistic is a weighted combination of chi-squares, weights hard to
estimate nonparametrically ((``literature has not presented methods
for ...'')). Without this reference distribution, testing for the null
directly is difficult. ((heller)) gives the asy null distribution
specifically for betahat estimated by mrc
method. % [[update: heller doesnt make modeling
% assumptions on the covariates, only on the estimation of betahat;
% maybe that is a benefit over testing risk function? ie is there a test
% for the mrc coefficients that doesnt require modeling covariates?]]
demler for assuming coefficents are esitmated by lda with gaussian
covariates. recently noted ((cite)) that asy null distribution remains
intractable for common estimation methods eg logistic regression.

((pepe 2013)) provides a more uniform approach to the problem of testing
 for zero difference between nested index aucs. The authors suggest a more
convenient often equivalent testing problem. The risk function for a
binary RV $\D$ based on a set of covariates $\W\in\mathbb{R}^p$,
$\risk[\W]:\mathbb{R}^p\to\mathbb{R}$, is the function
$\z \mapsto \P(\D=1 \mid \W=\z)$. Let $\W[1],\W[2]$ be two sets of
covariates and $\D$ binary. The authors show that the null of equal
AUCs of the risks,
\[
  \P(\risk[{\W[1],\W[2]}] < ... \mid D,...
\]
holds if and only if the risk functions are equal,
$\risk[{\W[1],\W[2]}]=\risk[{\W[1]}]$. Often the coefficient
estimation procedure is of secondary importance and the goal of
testing the null $\aucdiff=0$ is to test if certain additional
covariates improve discrimination. In this case, the test may be based
on the risks instead. Even if interest lies in testing for the
difference in AUCs where $\hat\beta,\hat\gamma$ are obtained through a
particular estimation procedure, e.g., logistic
regression,% Suppose it is needed to test
% $P(\beta^T(x,y)|D=0 < \beta^T(x,y)|D=1)=P(\gamma^Tx|D=0 <
% \gamma^Tx|D=1)$, obtained by LDA, logistic regression etc.
for many estimation procedures there is a monotone link connecting the limiting index to the risk,
e.g., the expit function. Since the AUC is invariant to monotone
transformations, the risk may still be used to test for a
difference. % Provided the coefficient
% estimation model is correct, the two models, reduced and full, are compatible under the
% null, since then $\star\beta=\star\gamma$, 
% so that testing if there is some monotone
% link $h$ such that
% $P(D=1|x,y)=h(\beta^T(x,y)),P(D=1|x)=h(\gamma^T(x))$, then the test is
% the same as
% $P(risk(x,y)|D=0 < risk(x,y)|D=1)=P(risk(x)|D=0 < risk(x)|D=1)$ so one
% may just test $risk(x,y)=risk(x)$. 

A drawback to this approach is it requires knowing the true risk
function. If the null distribution of $\aucdiffhat$ were available,
one might compare the discrimination of the indices $\t{\hat\beta} \W$ and
$\t{\hat\gamma} \W$, and possibly use the indices in practice, without
knowing the correct risk function. However, unless computing the null
distribution of the $\aucdiff$ calls for fewer modeling assumptions,
improved efficiency, or something else to recommend it, may as well
test risk functions.

We therefore only consider the alternative case ((ref)) in the remainder.

2. A second issue is that $\hat\beta,\hat\gamma$ are estimated from
the data so that the observations on which the U-statistic is based
are not IID. Non-degenerate U-statistics with estimated parameters are
typically still normal though estimation of the parameter in general affects
the asy distribution. This issue is addressed in the remainder.

((maybe mention that nolan-pollard papers address both problems in
almost the same setting (degree 2 ustats) but still need to
nonparametrically compute the liiting chi squared distribution for the
null))

\section{Method}


The usual approach to finding the asymptotic distribution of
a U-statistic, which we adopt, is to find an asymptotically equivalent IID mean, to which
the CLT can be applied.


For control and case distributions $F,G$ on $\mathbb{R}^p$ and vector $\beta$, denote the AUC of the index,$P(\beta' X<\beta' Y)$ for a control $X\sim F$ and independent case $Y\sim G$, by
\begin{align}
  \auc(F,G,\beta) &= \int\kernel{\t\beta x}{\t\beta y}dF(x)dG(y)
\end{align}
With this notation,
$\aucdiffhat=\auc(\hat\F,\hat\G,\hat\beta)-\auc(\hat\F,\hat\G,\hat\gamma)$.
We write each term as an IID sum, and later take the difference to
represent $\aucdiffhat$ as an IID sum. Decompose the centered estimate
$\auc(\hat\F,\hat\G,\hat\beta)- \auc(\F,\G,\beta)$ as a sum of two
terms, reflecting the two sources of estimation, the AUC estimation
and the coefficient
estimation%, the CDFs $\hat\F,\hat\G,$ and the parameter $\beta$.
\begin{align}
  &\auc(\hat\F,\hat\G,\hat\beta) - \auc(\F,\G,\beta)\\
  &=\auc(\F+\delta\F,\G+\delta\G,\beta+\delta\beta) - \auc(\F,\G,\beta+\delta\beta) \label{theory:term 1}\\
    &+ \auc(\F,\G,\beta+\delta\beta)-\auc(\F,\G,\beta)\label{theory:term 2}
\end{align}
Where $\delta\F=\hat\F-\F,$ etc.

term \eqref{theory:term 1}: As the function $\auc(\cdot,\cdot,\beta)$ is bilinear,
\begin{align}
  &\auc(\F+\delta\F,\G+\delta\G,\beta+\delta\beta) - \auc(\F,\G,\beta+\delta\beta)\\
  &=\auc(\delta\F,\G,\beta+\delta\beta)+\auc(\F,\delta\G,\beta+\delta\beta)+\theta(\delta\F,\delta\G,\beta+\delta\beta)
\end{align}
The third and final term in the sum is $o((\N)^{-1/2})$, % $\theta(\delta\F,\delta\G,\beta+\delta\beta)=o(n^{-1/2})$:
$$
\P(\theta(\delta\F,\delta\G,\beta+\delta\beta)>n^{-1/2}\epsilon) \le \P(\int d|\F_n-\F|(x) > n^{-1/4}\sqrt\epsilon)+\P(\int d|\G_n-\G|(x) > n^{-1/4}\sqrt\epsilon)\to 0
$$
by a DKW-type bound.
((Pf: By multivariate DKW (e.g., Serfling p.61), $\P(|\F_n-\F|_\infty > n^{-1/4})<c\exp(-c\sqrt n)$))
% ..((in case this approach doesnt work, can just cite nolan--pollard approach))

% The first two terms, partially expected and for fixed $\beta+\delta\beta$, are IID
% averages.  To account for the randomness in $\beta+\delta\beta$ expand in a Taylor series % if differentiable at $\beta$ ((always differentiable for $\beta\neq 0$?)),
% \begin{align}
%   ...
% \end{align}
% and if sufficiently smooth, can interchange derivtive and expectation
% \begin{align}
%   ...=0
% \end{align}
% since $\E(...)=0$ for all $\beta\neq 0$.

% ((generally not smooth at $\beta=0$, auc kernel jumps))

For fixed $\beta+\delta\beta$, the first two terms are centered IID averages
That the randomness in $\delta\beta$ is asymptotically negligible ((at this rate))
$$
\auc(\delta\F,\G,\beta+\delta\beta)+\auc(\F,\delta\G,\beta+\delta\beta)
=\auc(\delta\F,\G,\beta)+\auc(\F,\delta\G,\beta) + o((\N)^{-1/2})
$$
follows from empirical process theory, in particular stochastic equicontinuity of the process $\beta\to...$ ((maybe cite pollard)). %Consequently term 1 ((ref)) is % ((need to discuss derivative of hajek term at starred parameter being 0))))

Therefore,
\begin{align}
  % &\auc(\F+\delta\F,\G+\delta\G,\beta+\delta\beta) - \auc(\F,\G,\beta+\delta\beta)\\
  &\auc(\F+\delta\F,\G+\delta\G,\beta) - \auc(\F,\G,\beta) + o((\N)^{-1/2})\\
  &=-\frac{1}{\m}\sum_{i=1}^\m(1-\G(\t{\beta}\X[i])-\auc(\F,\G,\beta)) + \frac{1}{\n}\sum_{i=1}^\n(\F(\t{\beta}\Y[i])-\auc(\F,\G,\beta))+ o((\N)^{-1/2})\\
  &=\frac{1}{\N}\sum_{i=1}^{\N}\left(-\frac{\{\D[i]=0\}}{\P(\D=0)}(1-\G(\t{\beta}\W[i])-\auc(\F,\G,\beta)) + \frac{\{\D[i]=1\}}{\P(\D=1)}(\F(\t{\beta}\W[i])-\auc(\F,\G,\beta))\right) + o((\N)^{-1/2})
\end{align}
((last line doesnt maintain the $1/n$ rate? nvm it does--factor out
the clsas probability estimate and end up with the product of two
averages)) Known as the Hoeffding decomposition of $\aucdiff$, same as
the first von Mises derivative. Represents term \eqref{theory:term 1}
as an IID sum to which the CLT may be applied to get the asymptotic
distribution of $\aucdiff$ if term \eqref{theory:term 2} were
negligible, e.g., if $\hat\beta$ were not estimated (see ((ref below)
for additional scenarios when \eqref{theory:term 2} is
negligible)). The Delong approach in this situation is to estimate
$\F,\G$ using the empirical CDFs $\hat\F,\hat\G$, giving rise to the
standard Delong statistic for inference on $\aucdiff$.
\begin{align}
  ...
\end{align}



term \eqref{theory:term 2}: Assume
$\sqrt{n}(\hat\beta-\star\beta)\to 0$ in probability,
$\beta\mapsto\auc(\F,\G,\beta)$ is differentiable at $\star\beta$. Let
the function $\infl[\hat\beta]$ represent the estimator $\hat\beta$ as
an IID mean
\begin{align}
  \hat\beta-\star\beta=(\N)^{-1}\sum_{i=1}^{\m+\n}\infl[\hat\beta](\W[i]) + o((\N)^{-1/2})
\end{align}
i.e., $\infl[\hat\beta]$ is an influence function for the $\hat\beta$. Then \eqref{theory:term 2} is
\begin{align}
  &\auc(\F,\G,\beta+\delta\beta)-\auc(\F,\G,\beta)  \\
  &=(\hat\beta-\star\beta)\frac{\partial}{\partial\beta}\auc(\F,\G,\beta) + o_P((\N)^{-1/2})\\
  &=\frac{\partial}{\partial\beta}\auc(\F,\G,\beta)(\N)^{-1}\sum_{i=1}^{\m+\n}\infl[\hat\beta](\W[i]) + o_P((\N)^{-1/2}).
\end{align}

Putting the two parts together,
\begin{align}
  &\auc(\hat\F,\hat\G,\hat\beta) - \auc(\F,\G,\beta)\\
  &=\frac{1}{\m+\n}\sum_{i=1}^{\m+\n}\left(-\frac{\{\D[i]=0\}}{\P(\D=0)}(1-\G(\t{\star\beta}\W[i])-\auc(\F,\G,\star\beta)) + \frac{\{\D[i]=1\}}{\P(\D=1)}(\F(\t{\star\beta}\W[i])-\auc(\F,\G,\star\beta))\right) \\
  &+ \frac{\partial}{\partial\beta}\auc(\F,\G,\beta)\sum_{i=1}^{\m+\n}\infl[\hat\beta](\W[i]) + o_P((\N)^{-1/2}).
\end{align}
((maybe combine into one sum))

Proposition.
Assumptions: available influence function. ((continuous
CDFs?)) $P(\D=0) \in (0,1)$. non-degeneracy condition: probability
limits for $\hat\beta$, and
... differentiability of $\auc$ at
$\star\beta,\star\gamma$.
Assertion: $(\N)^{-1}(\auc(\hat\F,\hat\G,\hat\beta)-\auc(\F,\G,\star\beta))$ is asymptotically normal with mean zero and variance ((variance of a term in the IID mean)). This variance may be consistently estimated as $\sqrt{\N}$ times the sample variance of ((ref IID mean)).

take the difference with the same representation of another estimator, $\auc(\hat\F,\hat\G,\hat\gamma)$, to obtain an IID representation of $\aucdiffhat$.
corollary.
Assumptions: Assumptions of ((ref prop)) apply to $\hat\beta,\hat\gamma$ both, also $\star\beta\neq\star\gamma$. Then $(\N)^{-1}(\aucdiffhat-\aucdiff)$ is asymptotically normal with mean zero and variance given by a term in the difference of IID means ((ref)). This variance may be consistently estimated as ... ((as in prop))

REMARKS

- Benefits of approximating by an iid sum:

1. As above, can de-couple the two terms of the difference $\aucdiff$ and treat
estimation the auc of an index using an estimated index.

2.  $\hat\theta$ is an IID sum, and the sd estimate is the empirical
estimator. This is itself an estimate of $\sqrt{\N}\hat\theta$, not
$\hat\theta$. So etimated parameters in ((ref single sum)) are
asymptotically negligible as long as the parameter estimates are
consistent and dependence is continuous. of course may affect
efficiency of asy convergence. %  The influence function may contain
% nuisance parameters as long as it depends on them continuously and
% consistent estimators are available.
((Just as the standard Delong estimator estimates the conditional CDFs
$\F,\G$ using empirical CDFs))

3. The methods apply not only to testing indexes based on nested data
sets, but more generally to a comparison of any correlated AUCs with
index coefficients estimated from the data, eg LDA versus logistic.

- It isn't needed that $\hat\beta$ be estimated by a correctly specified
model, only that it has some probability limit at the parametric rate.
Though procedure for obtaining the estimate $\hat\beta$ and the
associated influence function $\infl$ often involve some parametric
assumptions, we still term the procedure described here as
``non-parametric'' since the estimate ((ref)) is valid under
misspecification. Whatever the estimation procedure is it will be
known to the analyst, so that an influence function may be chosen, if
one exists.

% ((all of these o(n) expressions should be o(m+n) ))...
% ((influence function should be average not sum))

% adding the two parts term-wise gives an iid representation of $\auc(\hat\beta)$:


- What goes wrong under the null? % need to know this formally in order to
% state theorem. 
\eqref{theory:term 1} hajek parts will be the same if
$\star\beta=\star\gamma$, so \eqref{theory:term 1} will be
$o(n^{-1/2})$.  If the probability limit of $\hat\beta$ and
$\hat\gamma$ are the same, then in term \eqref{theory:term 2} the
derivatives are the same. In many situations where the index is
derived from a well-specified model the derivative is $0$ for at least
one of the two AUCs being differenced. In that case also
\eqref{theory:term 1} will be $o(n^{-1/2})$. just a sufficient
condition, possible that in e.g. a logit model both full and reduced
are misspecified, and then the limit is normal. ((check))((also check
against demler 2017 iff conditions for degeneracy))


((use of starred parameter inconsistent in this section... drop it
here in favor of $\delta\beta$ ?))

\section{Examples}

\subsection{No effect of coefficient estimation}
In the ordinary course, the coefficient estimation can be ignored in
computing the index of an auc iff the derivative ((ref)) is 0. For the
difference of two AUCs $$ ... $$, the derivative of each must usually be 0 at
the respective coefficient probability limits,
$\star\beta,\star\gamma$.  In that case the usual delong statistic
may be used, provided of course the AUCs are distinct ((ref intro)).


\subsubsection{AUC}

Examples where the coefficient estimation may be ignored in estimating the AUC of an index.

Example. estimator: mrc, covariate restrictions:
none/nonparametric. The maximum rank correlation method of computing
the coefficients is $$ ... .$$ The method is non-parametric. By
construction the empirical AUC is stationary at the coefficient
estimates, and ((under regularity conditions)) the AUC is stationary
at the probability limits, as well. 

The following proposition ((highlighted by the work of Pepe)) furnishes other examples.

Proposition:

Given RVs $(\W,\D)$, $\W$ continuous, $\D$ binary% indicating class membership of $\W$
 
1. The ROC curve of predicting $\D$ based on a real function of $\W$
is maximized pointwise over all such functions by the likelihood ratio
((...)), equivalently, the risk of $\D$ based on $\W$, $\risk[{\W}](\cdot)$

2. The AUC of a real function $f$ of $\W$ is maximal iff $f$ is
has the same conditional distribution given $\D$ as an increasing function of the risk of $\D$ based on $\W$, i.e., there is a strictly
increasing function $h:\mathbb{R}\to\mathbb{R}$ such that
$P(\risk(x)<\xi|\D=i)=P(h\circ f(\W)<\xi|\D=i)$ for all $\xi\in\mathbb{R}$ and
$i=0,1$.
% 3. if deriv is concave,

3. Assuming as above the derivative of the AUC is smooth at
$\star\beta$, can use delong statistic if the index is related to risk by
an increasing function. ((maybe introduce notation $~$, then can say $\t{\star\beta}\W \sim \risk(\W)$))

Proof. 1. Is the neyman pearson lemma, as pointed out by ((pepe? check
if she did it first. swets.)). Let FPR value $\alpha\in (0,1)$ be
given. Viewing $\D$ as a parameter, the most powerful level $\alpha$
test of the null $\D=0$ versus the simple alternative $\D=1$ rejects
for large values of the likelihood ratio of ((x,g)). Therefore the
value of the ROC curve of the likelihood ratio at $\alpha$, which is
the power of the Neyman-Pearson test, is maximal. Since the ROC curve is the same for
incrasing functions of the likelihood, and ((show likelihood is expit
of risk)), the same holds of the risk. 2. Though markers not related
by an increasing function may have the same AUC, since the ROC curve
of the risk is maximal, an index with the same AUC must have the same
roc curve, which does imply the index has the same conditional
distributions as an increasing function of the risk.

Example. coefficient estimator: irrelevant, covariate restrictions: A
single covariate. When there is a single covariate, $p=1$, the $\beta$
in the ((AUC formula)), for $\beta\neq 0$, cancels and the requirement is
simply that the risk be increasing in the sole covariate, i.e., that
the covariate or its negation be a risk
factor.% ((ref simulations in demler 2017, where reduced model has a
% single covariate)).

Example. Parametric models where index is monotonically related to the
risk. The derivative will vanish in smooth parametric models under
which the index is monotonically related to the risk function.% is
 % sufficient. %may also be necessary if deriv is concave.


sub-Example. coefficient estimator: binary response MLE, covariate
restrictions: glm link. A prominent example where the index is an
increasing function of the risk is the index model for a binary
response:
\begin{align}
  \P(\D=1) = \h(\t\beta \w), \beta\in\mathbb{R}^p
\end{align}
The function $\h$ is strictly increasing, such as a probit link,
logistic link, identity, etc.

sub-Example. coefficient estimator: lda (homoskedastic), covariates:
multivariate gaussian (just with gaussian data? conjecture in demler
2012 re elliptic distributions. maybe expand on this example in the
misspecified lda section.) ((lda is collapsible more generally, but
risk may not be an increasing function of the index without the
gaussian assumption))

((maybe give proof here, or can mix it in with longer example below))


sub-Example. coefficient estimator: lda (heteroskedastic), covariates:
independent exponential family data mean parameterized

Let $\W[i]\mid \D=j$ have density
$h_i(x_i)\exp(\theta_{ij}'t_i(x_i)-A_{ij})$. If the covariates are
independent, the likelihood ratio is then
\begin{align}
 \frac{f(x\mid \D=1)}{f(x\mid \D=0)} \sim \sum_{i=1}^n(\theta_{i1}-\theta_{i0})^tx_i
\end{align}
((notation: n, also x's.))

If LDA is used to estimate $\beta$, then
\begin{align}
  \hat\beta \to_p ...
\end{align}
and the index at probability limit is $...$. If 1. the covariates have the same population variances, say $\pi_0A_0''+\pi_1A_1''$, and 2. the parameter $\theta_{ij}$ is the mean $A_{ij}$, then
\begin{align}
  \beta x \sim \sum_{i=1}^n(A_{i1}'-A_{i0}')x_i \sim \risk (x)
\end{align}


With gaussian data as in 2 but heteroskedastic, or heteroskedastic exponential family as in
2 but not independent, the derivative of $\star\auc$ need not vanish. [[ref
heteroskedastic lda example below.]]

\subsubsection{difference of AUCs}

For a non parametric estimator like MRC there is no difficulty. ((cite
heller)) Each estimation procedure leads to a vanishing derivative.

also no problem when one coefficient is estimated by a well-specified
parameric estimator and the other by a non-parametric estimator, or e.g.
the single covariate case where there is effectively no estimator. % full
% model well-specified and reduced model a single covariate 
((cite demler 2017 simulation here))


% To apply this example as justification for inference based on the
% standard Delong statistic requires that both AUC models be
% well-specified. In the case of comparing a full $...$ to a reduced
% model $...$.


When both coefficient vectors being compared are modeled
parametrically. consider specifically nested ((binary response)) models.
break up into 3 cases. If neither the full model nor the reduced model
is well spcified, $...$, there is no reason to expect the derivative
to vanish and in general coefficient estimation must be accounted
for. When the reduced model is well specified, then comparison with a
superset of the covariates will generally lead to the null situation,
i.e., a degenerate U-statistic ((ref intro)).

% if the full well-specified, reduced not, vice
% versa, and both misspecified. If neither is correctly specified,
% require accounting for derivative term. if reduced model is correct,
% then full model is.
Finally, consider the situation that the full model is correct,
reduced need not be.  eg when the fuller model contains a superset of
the model covariates, and the reduced model a strict subseet. This
isutation is common ((give citations to simulations/data anlayses)) as in many cases correctness of 
the full model,
\[
\]
implies the reduced model usually cannot be correct
\[
\]
In this situation the derivative term will be nonzero and must be
accounted for.
% In the case of comparing a
% correctly specified full model $\P(D=1)=...$ to a reduced model $...$,
For the binary response model, the requirement is that the marginalization does not break the model,
equality holds in ((ref above)) ((maybe cite bridge distribution
paper.)) Some examples where this requirement holds are:
% and it often
% will not be the both will be well specified. excpet collapsible models,
% where the derivative is 0 in both models and therefore--assuming
% correct specification--can just use the basic delong stat, ignoring
% estimation of aprameters.

% need to check derivative 0 and also model assumptions.


% 2. full model well specified, implying reduced model also well
% specified. ie the true covariates only need be a subset of those
% available.

- probit regession with gaussian covariates.

- gaussian lda model ((cite demler here, mahalanobis distance connection)), heteroskedastic lda model

- the logistic view of LDA: logistic regression with a gaussian mixture as covariates in the special case that beta is the lda/malanobis dist beta. This example is almost the same as the homoskedastic LDA example, since [[ref fisher lda display]] the posterior probabilities are given by ... . 


% adam at applestore cambridge, say chris at customer rleations said to call

% examples where derivative is 0:
% 1. single covariate. with a single predictor the derivative will often be zero, bc the risk
% (where the derivative of the auc is 0), $r(x)$, is often a monotonic
% function of $x$, and
% $P(\beta x_0 < \beta x_1) = P(x_0 < x_1) = P(r(x_0) < r(x_1))$ in this
% case. Demler 2017 uses a single covariate in the reduced model, so
% their example doesn't say much.
% % 2. exponential family (see notes)

convergence to 0 of the non-delong part can be slow. even on simple
data like iid gaussian covariates with a null logistic model.


\subsection{must account for beta estimation}
Next we describe situations where must account for the $\beta$ parameter estimation
include: misspecification in one of the above situations ((can view
ths proposed approach as adding robustness)). or, correct specified
but nonzero derivative still.

\begin{example} heteroskedastic gaussian lda.  Suppose Gaussian linear
  discriminant analysis is applied to estimate beta but possibly misspecified
  in that the two classes may not have the same covariance. The model
  is
  \begin{align}
    \W | \D=i \sim \F_i=N_p(\mu_i,\Sigma_i)\\
    \P(\D=1)=1-P(\D=0)=\pi_1
  \end{align}
  The LDA parameter estimation procedure is to base class membership on the sign of $\beta'x$
  ((ignore intercepts without loss)), where
  \begin{align}
    \hat\beta=...\\
    \hat\mu=...\\
    \hat\Sigma = ...\\
  \end{align}
  the LDA parameter estimates ((ref above)), which assume a common variance for the two classes, tend in probability to
  \begin{align}
    \star\beta &=...\\
    \star\Sigma &= ...\\
  \end{align}
  The AUC and its derivative at the starred parameters are
  \begin{align}
    \auc(\F,\G,\star{\beta}) &=...\\
    \auc'(\F,\G,\star{\beta}) &= ...
  \end{align}
  The derivative is $0$ iff ((eigenvector condition)). In terms of the
  normal means and variances, this condition is ((...)). 2 examples:
  1. independent data, ((...)), 2. proportional covariance
  matrices. The first is already implied by the general exponential
  family result [[ref above]] but not the second as the observations
  are not independent.

  show that derivative term can be unbounded. % must consider also
  % variance of infl function ie inverse fisher information, i.e., the
  % components of  % It is possible for $\auc'(\F,\G,\star{\beta})$ to have large
  % % components which are countered by variance of the parameter
  % % estimates. 
  When $\Sigma\approx\star\sigma$ [[not defined yet
  $\Sigma=\Sigma_0+\Sigma_1$]], the the influence function is
  approximately $O(|\Sigma|^{-1/2})$, the root of the inverse Fisher
  information, and $ \auc'(\F,\G,\star{\beta})$ is approximately
  $O(|\Sigma|^{1/2})$, so that the product, giving the entire
  non-Delong term, is approximately $O(1)$. ((maybe give quantitative bound on derivative in terms of class imbalance)) Nevertheless it is
  possible to push the proportion so that the entire non-Delong term
  $(\star{\Sigma})^{-1} \auc'(\F,\G,\star{\beta})$ has large
  components.

  Let
  \begin{align}
    \Sigma_0 = ...
  \end{align}
  Then $(\star{\Sigma})^{-1} \auc'(\F,\G,\star{\beta}) \to \pm \infty \mathbbm{1}$ ((write out?)) as $\pi_0 \to 1$ and $\epsilon\to 0$ simultaneously. One therefore expects that under this scenario inference based on the Delong estimator will be faulty, as verified by simulation in ((ref simulation section)). ((possible to give result so that the sign of the derivative term is controlled, so that delong can be forced either to low fpr or low power?))
  
\end{example}



\section{Simulation}


\section{Discussion}
approach extends straightforwardly to other differentiable functions
of the data $f(x)$, not just the linear combination.

extends to discrete covariates (modified auc kernel)

\end{document}

