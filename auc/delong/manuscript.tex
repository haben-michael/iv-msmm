\documentclass[12pt]{article}
\usepackage{amsthm}
\usepackage{bbm}
\usepackage{amssymb}
\usepackage{mathtools}
\mathtoolsset{showonlyrefs,showmanualtags}
\usepackage{etoolbox}
% \usepackage{booktabs}
% \usepackage{url}
%\usepackage{setspace}
\usepackage[margin=1in]{geometry}
% \usepackage{authblk}
\usepackage{natbib}
% \usepackage[page]{appendix}
\usepackage[nomarkers,nolists]{endfloat}
\usepackage{graphicx}
% \usepackage{tikz}
% \usetikzlibrary{calc}
\usepackage{subfig}
\usepackage{enumitem}
% \usepackage{array} % tables with fixed width and alignment
\DeclareMathOperator{\AUC}{AUC}
\DeclareMathOperator{\V}{Var}
\DeclareMathOperator{\cov}{Cov}
\DeclareMathOperator{\corr}{Corr}
\DeclareMathOperator{\sd}{sd}
\newcommand{\E}{E}
\renewcommand{\P}{P}
\newcommand{\cind}{\perp \!\!\! \perp}
\newcommand{\X}[1][]{X_{0#1}}
\newcommand{\Y}[1][]{X_{1#1}}
\newcommand{\Z}[1][]{X_{#1}}
\renewcommand{\star}[1]{{#1}^\ast}
\newcommand{\F}{F}
\newcommand{\G}{G}
\newcommand{\D}{D}
\newcommand{\m}{m}
\newcommand{\n}{n}
\newcommand{\risk}{\rho}
\newcommand{\auc}{\theta}
\newcommand{\betastar}{\beta_0}
\newcommand{\aucdiff}{\Delta\text{AUC}}
\newcommand{\aucdiffhat}{\hat{\Delta\text{AUC}}}
\newcommand{\infl}{\phi}
\newcommand{\h}{h}
\newcommand{\termb}{term 2 }
\newtheorem{theorem}{Theorem}
\newtheorem{proposition}[theorem]{Proposition}
\newtheorem{lemma}[theorem]{Lemma}
\newtheorem{corollary}[theorem]{Corollary}
\theoremstyle{definition}
\newtheorem{example}{Example}%[section]
\makeatletter
% \def\input@path{{input/}{figs/}}
% \graphicspath{{./figs/}}
\newtoggle{commenttoggle}
\togglefalse{commenttoggle}
\newcommand{\comment}[1]{
  \iftoggle{commenttoggle}{
    {\normalsize{\color{red}{ #1}}\normalsize}
  }
  {}
}
\title{Nonparametric estimation of the auc of an estimated index}
% \author[1]{Haben Michael}
% \author[2]{Lu Tian}
% \affil[1]{University of Massachusetts}
% \affil[2]{Stanford University}
\date{}

\begin{document}
\maketitle

Abstract. We describe a nonparametric method of estimating the AUC of
an index $\beta'x$ when $\beta$ is estimaed from the same data, with a
focus on nonparametric estimation of the difference of the AUCs of two
distinct indices.

\section{Introduction}

measuring the discrimination of biomarkers using the auc. $\aucdiff$ to
measure the difference in discrimination between two markers. often
one is based on a subset of covariates of the other marker.

demler 2017 $\Delta$AUC is one of the most widely used measures of discrimination.((should be ``difference'' in discrimination I think?))

\section{Background/setting}


% difference of 2 aucs.
% may be viewed as a U-statistic. 
Let
\begin{align}
  \X[1],\ldots,\X[\m] \sim \F,  IID,   \Y[1],\ldots,\Y[\n] \sim \G, IID
\end{align}

Based on this data, analyst obtains coefficients $\beta,\gamma$. May be that one is based on a subset of those coefficients on which the other is based.

The statistic $\aucdiffhat$ is
\begin{align}
  \frac{1}{\m\n}\sum_{i,j}\{\hat\beta'\X[i]<\hat\beta'\Y[j]\}
  -  \frac{1}{\m\n}\sum_{i,j}\{\hat\gamma'\X[i]<\hat\gamma'\Y[j]\} 
\end{align}


An explicit probabilty model may not be specified, and often the two
estimation methods imply inconsistent models, e.g., logistic models
with nonzero covariates omitted from the reduced model. ((this last example only
an issue under the alternative)) Nevertheless inference is sought,
particularly 1. whether the difference in the AUCs of the two markers
$\hat\beta' x$ and $\hat\gamma'x$, in some limiting sense ((introduce
starred parameters here?)) is nonzero, and if so 2. the magnitude of
the difference.

For fixed $\hat\beta,\hat\gamma$, this statistic may be viewed as a
two-sample U-statistic with kernel
$(x,y)\mapsto \{\hat\beta'x<\hat\beta'y\} -
\{\hat\gamma'x<\hat\gamma'y\}$.

Two complications for the basic u-stat theory.

1. under the null of no diff, asy distribution is a weighted
combination of chi-squares, weights hard tocompute etc. pepe 2013
helps to resolve the problem of testing the index aucs for significant
difference. Let the risk function based on a set of covariates $\X$ be
denoted $\risk_{\X}(\cdot)=P(D=1\mid \cdot)$. Let $X,X',...$. They show that the null of equal AUCs of ...,
\[
  P(\risk_{...}(\X,\X') < \risk(\Y,\Y'))  = P(\risk_{...}(\X)<\risk(\Y))
\]
holds if and only if the risk functions are equal,
$\risk_{\X,\X'}=\risk_{\X}$. Often the estimation procedure is of secondary importance and the goal of testing the null $\aucdiff=0$ is to test the the additional covariates improve discrimination. In this case, the test may be based on the risks instead. Even if an analyst is interested specifically in testing for the difference in AUCs where $\hat\beta,\hat\gamma$ are obtaiend through a particular procedure, e.g., logistic coefficients. % Suppose it is needed to test
% $P(\beta^T(x,y)|D=0 < \beta^T(x,y)|D=1)=P(\gamma^Tx|D=0 <
% \gamma^Tx|D=1)$, obtained by LDA, logistic regression etc.
there will often be a monotone link connecting the index to the risk,
e.g., the expit function. Since the AUC is invariant to monotone
transformations, the risk may still be used to test for a
difference. The two models, reduced and full, are compatible under the
null, since then $\star\beta=\star\gamma$, provided the coefficient
estimation model is correct.% so that testing if there is some monotone
% link $h$ such that
% $P(D=1|x,y)=h(\beta^T(x,y)),P(D=1|x)=h(\gamma^T(x))$, then the test is
% the same as
% $P(risk(x,y)|D=0 < risk(x,y)|D=1)=P(risk(x)|D=0 < risk(x)|D=1)$ so one
% may just test $risk(x,y)=risk(x)$. 

A drawback to this approach is it requires knowing the true risk
function. one would be able to obtain the indices $\beta^T$ and
$\gamma^T$ and compare the discrimination, and use the indices in
practice, without knowing the correct risk function. However, unless
computing the null distribution of the $\aucdiff$ calls for fewer modeling assumptions, improved efficiency, or something else to recommend it, may as well test risk functions.


literature: heller gives the asy null distribution specifically for betahat
estimated by mrc method.  [[update: heller doesnt make modeling
assumptions on the covariates, only on the estimation of betahat;
maybe that is a benefit over testing risk function? ie is there a test
for the mrc coefficients that doesnt require modeling covariates?]]
demler for lda with gaussian covariates ((no benefit over pepe
approach here since parametric assumptions imposed)). recently noted
((cite)) that asy null distribution remains intractable for common
estimation methods eg logistic regression.


therefore only consider the alternative here.

2. second issue is that betahat is estimated from the data so that the
observations to which the u-statistic is applied are not
iid. Non-degenerate u-stat with estimated parameters is typically
still normal, though demonstration requires viewing the ustat as a
process indexed by random index beta. estimation of the parameter in
general affects the asy distribution. address this issue in the remainder`.

\section{Theory}

\subsection{IID sum}
the approach, which we adopt, is to rewrite more complicated
estimators as asymptotically equivalent iid sums, amenable to
conventional analysis. benefits of approximating by an iid sum:
1. can de-couple the two
parts of the difference and just focus on the estimating the auc of a
an index estimated (possibly misspeciifed model) from the data.
2.  $\hat\theta$ is an IID sum, and the sd estimate
is the empirical estimator. This is itself an estimate of
$\sqrt{n}\hat\theta$, not $\hat\theta$. So etimated
parameters are OK (take taylor expansion) as long as the parameter
estimates are consistent and dependence is continuous. of course may affect efficiency of asy convergence.
3. produce routines to reduce auc(beta.hat) to an iid sum,
whatever the data is. can apply this separately to the full and
reduced data sets, but also applies more generally to a comparison of
any correlated aucs with parameters estimated from the data, eg
lda versus logistic.



For CDFs $F,G$ on $\mathbb{R}^p$ and vector $\beta$ define the notation for the AUC of the index  $P(\beta' X<\beta' Y), X\sim F,Y\sim G,X\cind Y,$ 
\begin{align}
  \auc(F,G,\beta) = \int\{\beta'x<\beta'y\}dF(x)dG(y).
\end{align}
In this notation,
$\aucdiffhat=\auc(\hat\F,\hat\G,\hat\beta)-\auc(\hat\F,\hat\G,\hat\gamma)$.
We write each term as an IID sum, and later take the difference to
represent $\aucdiff$ as an IID sum. Decompose the centered AUC
$\auc(\hat\F,\hat\G,\hat\beta)- \auc(\F,\G,\beta)$ as a sum of two terms, reflecting
the two sources of estimation%, the CDFs $\hat\F,\hat\G,$ and the parameter $\beta$.
\begin{align}
  &\auc(\hat\F,\hat\G,\hat\beta) - \auc(\F,\G,\beta)\\
  &=\auc(\F+\delta\F,\G+\delta\G,\beta+\delta\beta) - \auc(\F,\G,\beta+\delta\beta) \label{theory:term 1}\\
    &+ \auc(\F,\G,\beta+\delta\beta)-\auc(\F,\G,\beta)\label{theory:term 2}
\end{align}
Where $\delta\F=\hat\F-\F,$ etc.

term \eqref{theory:term 1}: The function $\auc(\cdot,\cdot,\beta)$ is bilinear,
\begin{align}
  &\auc(\F+\delta\F,\G+\delta\G,\beta+\delta\beta) - \auc(\F,\G,\beta+\delta\beta)\\
  &=\auc(\delta\F,\G,\beta+\delta\beta)+\auc(\F,\delta\G,\beta+\delta\beta)+\theta(\delta\F,\delta\G,\beta+\delta\beta)\\
    &=-\frac{1}{\m}\sum_{i=1}^\m(1-\G(\hat\beta'\X[i])-\auc(\F,\G,\hat\beta)) + \frac{1}{\n}\sum_{i=1}^\n(\F(\hat\beta'\Y[i])-\auc(\F,\G,\hat\beta))+ o(n^{-1/2})
\end{align}

The approximation in the third
inequality,$\theta(\delta\F,\delta\G,\beta+\delta\beta)=o(n^{-1/2})$: $\P(\sqrt{n}\theta(\delta\F,\delta\G,\beta+\delta\beta)>\epsilon) \le \P(n^{1/4}\int d|\F_n-\F|(x) > \sqrt\epsilon)+\P(n^{1/4}\int d|\G_n-\G|(x) > \sqrt\epsilon)$ ...((in case this approach doesnt work, can just cite nolan--pollard approach))

For fixed $\hat\beta$, the sums in [ref] are the Hoeffding
decomposition of $\aucdiff$. Same as the first von Mises
derivative. For fixed $\hat\beta$, represents term \eqref{theory:term
  1} as an IID sum to which the CLT may be applied to get the
asymptotic distribution of $\aucdiff$ if term \eqref{theory:term 2}
were negligible, e.g., if $\hat\beta$ were not estimated. The Delong
approach in this situation is to estimate $\F,\G$ using the empirical
CDFs $\hat\F,\hat\G$, giving rise to the standard Delong statistic for
inference on $\aucdiff$.

term \eqref{theory:term 2}: Assume $\sqrt{n}(\hat\beta-\betastar)\to 0$ in probability, $\beta\mapsto\auc(\F,\G,\beta)$ is differentiable at $\betastar$. Let the function $\infl$  represent the estimator $\hat\beta$ as an IID sum
\begin{align}
  \hat\beta-\betastar=\sum_{i=1}^{\m+\n}\infl(\Z[i]) + o(n^{-1/2})
\end{align}
i.e., $\infl$ is an influence function for the $\hat\beta$. Then \eqref{theory:term 2} is
\begin{align}
  &\auc(\F,\G,\beta+\delta\beta)-\auc(\F,\G,\beta)  \\
  &=(\hat\beta-\betastar)\frac{\partial}{\partial\beta}\auc(\F,\G,\beta) + o_P(n^{-1/2})\\
  &=\frac{\partial}{\partial\beta}\auc(\F,\G,\beta)\sum_{i=1}^{\m+\n}\infl(\Z[i]) + o_P(n^{-1/2}).
\end{align}
It isn't needed that $\hat\beta$ be estimated by a correctly specified
model, only that it has some probability limit at the $\sqrt{n}$
rate. The influence function may contain nuisance parameters as long
as it depends on them continuously and consistent estimators are
available. Though procedure for obtaining the estimate $\hat\beta$ and
the associated influence function $\infl$ often involve some
parametric assumptions, we still term the procedure described her as
non parametric since the model for $\hat\beta$ may be misspecified,
and the whatever the estimation procedure is it will be known to the
analyst, so that an influence function may be chosen, if one exists.


What goes wrong under the null? If the probability limit of
$\hat\beta$ and $\hat\gamma$ are the same, then in term
\eqref{theory:term 2} the derivatives are the same. if well specified
may be a transofmration of the risk and therfore stationary point,
derivatives will both vanish and \eqref{theory:term 2} will be
$o(n^{-1/2})$. What about term \eqref{theory:term 1}?

adding the two parts term-wise gives an iid representation of $\auc(\hat\beta)$:

take the difference with the same representation of another estimator, $\auc(\hat\gamma)$, to obtain an iid representation of $\aucdiffhat$:


\section{Examples}

\subsection{No effect of beta estimation}
In the ordinary course, betahat estimation can be ignored in computing
the index of an auc iff \termb is 0. in the case of $\aucdiff=...$, need deriv must ordinarily
 be 0 at both betahat and gammahat, in which case the usual delong statistic may
 be used.

 \termb will be 0 in many well-specified models for beta estimation
 due if the index is monotonically related to the risk function.%  is
 % sufficient. %may also be necessary if deriv is concave.

 proposition:

Given RVs $(\X,\D)$, $\D$ binary etc.
 
1. The roc curve is maximized pointwise over all real functions of $\X$ by the likelihood ratio, equivalently, the risk of $\D$ based on $\X$ ((define this phrase))

2. The auc of a real function of $\X$ is maximal iff the function is an increasing function of the risk
[at least if distributions assumed continuous] i.e., there is a
strictly increasing function $f:\mathbb{R}\to\mathbb{R}$ such that
$P(risk(x)<x_0|\D=i)=P(f(\beta' x)|\D=i)$ for all $x\in\mathbb{R}$ and
$i=1,2$.
% 3. if deriv is concave,

3. Assuming as above the derivative of the AUC is smooth at
$\betastar$, can use delong statistic if index is related to risk by
an increasing function.[more precisely: index has the same conditoinal distributions given binary status as an
increasing function of the risk ((at least if distributions are continuous))

Proof. 1. is neyman pearson lemma, as pointed out by ((pepe? check if she did it first. swets.)). viewing
$\D$ as a parameter, the most powerful test of the null $\D=0$ versus
the simple alternative $\D=1$ rejects for large values of the
likelihood ratio of ((x,g)). Therefore the ROC curve of the likelihood
ratio is maximal at each point. Since the ROC curve is the same for
incrasing functions of the likelihood, and ((show likelihood is expit
of risk)), the same holds of the risk. 2. Though markers not related
by increasing function may have the same auc, however since the roc
curve of the risk is maximal, an index with the same auc must have the
same roc curve, which does imply the index is distributionally equal
to an increasing function of the risk.

Example. A prominent example where the index is an increasing function of the
risk is the index model for a binary response:
\begin{align}
  \P(\D=1) = \h(\beta'x), \beta\in\mathbb{R}^p
\end{align}
The function $\h$ is strictly increasing, such as a probit link,
logistic link, identity, etc. In the $p=1$ case the $\beta$ cancels
and the requirement is simply that the risk be increasing in the sole
covariate, i.e., that the covariate or its negation be a risk factor.

However, the $\aucdiff$ consists of 2 aucs, so to apply this example
as justification for inference based on the standard Delong statistic
requires that both AUC models be well-specified. In the case of
comparing a correctly specified full model $\P(D=1)=...$ to a reduced
model $...$, the requirement is that the marginalization does not
break the model:
\begin{align}
\int ...
\end{align}
((maybe cite bridge distribution paper.)) Some
examples where this requirement holds are:
% and it often
% will not be the both will be well specified. excpet collapsible models,
% where the derivative is 0 in both models and therefore--assuming
% correct specification--can just use the basic delong stat, ignoring
% estimation of aprameters.

1. probit regession with gaussian covariates.

2. fisher lda (homoskedastic with gaussian covariates) (just with gaussian data? conjecture in demler 2012 re elliptic distributions. maybe expand on this example in the misspecified lda section.) ((lda is collapsible more generally, but risk may not be an increasing function of the index without the gaussian assumption)) 

((maybe give proof here, or can mix it in with longer example below))

This example was given by Demler 2011 (or 2012?)

3. the logistic view of the last example: logistic regression with a gaussian mixture as covariates in the special case that beta is the lda/malanobis dist beta. This example is almost the same as the homoskedastic LDA example, since [[ref fisher lda display]] the posterior probabilities are given by ... . 

4. lda (heteroskedastic) with independent exponential family data mean parameterized

Proof. Let $\Z[i]\mid \D=j$ have density $h_i(x_i)\exp(\theta_{ij}'t_i(x_i)-A_{ij})$. If the covariates are independent, the likelihood ratio is then
\begin{align}
 \frac{f(x\mid \D=1)}{f(x\mid \D=0)} \sim \sum_{i=1}^n(\theta_{i1}-\theta_{i0})^tx_i
\end{align}
((notation: n, also x's.))

If LDA is used to estimate $\beta$, then
\begin{align}
  \hat\beta \to_p ...
\end{align}
and the index at probability limit is $...$. If 1. the covariates have the same population variances, say $\pi_0A_0''+\pi_1A_1''$, and 2. the parameter $\theta_{ij}$ is the mean $A_{ij}$, then
\begin{align}
  \beta x \sim \sum_{i=1}^n(A_{i1}'-A_{i0}')x_i \sim \risk (x)
\end{align}

With gaussian data as in 2 but heteroskedastic, or heteroskedastic exponential family as in
2 but not independent, the derivative of $\star\auc$ need not be 0. [[ref
heteroskedastic lda example below.]]

% adam at applestore cambridge, say chris at customer rleations said to call

% examples where derivative is 0:
% 1. single covariate. with a single predictor the derivative will often be zero, bc the risk
% (where the derivative of the auc is 0), $r(x)$, is often a monotonic
% function of $x$, and
% $P(\beta x_0 < \beta x_1) = P(x_0 < x_1) = P(r(x_0) < r(x_1))$ in this
% case. Demler 2017 uses a single covariate in the reduced model, so
% their example doesn't say much.
% % 2. exponential family (see notes)

\subsection{must account for beta estimation}
Next we describe situations where must account for the $\beta$ parameter estimation
include: misspecification in one of the above situations ((can view
ths proposed approach as adding robustness)). or, correct specified
but nonzero derivative still.

\begin{example} heteroskedastic gaussian lda.  Suppose Gaussian linear
  discriminant analysis is applied to estimate beta but possibly misspecified
  in that the two classes may not have the same covariance. The model
  is
  \begin{align}
    \Z | \D=i \sim \F_i=N_p(\mu_i,\Sigma_i)\\
    \P(\D=1)=1-P(\D=0)=\pi_1
  \end{align}
  The LDA parameter estimation procedure is to base class membership on the sign of $\beta'x$
  ((ignore intercepts without loss)), where
  \begin{align}
    \hat\beta=...\\
    \hat\mu=...\\
    \hat\Sigma = ...\\
  \end{align}
  the LDA parameter estimates ((ref above)), which assume a common variance for the two classes, tend in probability to
  \begin{align}
    \star\beta &=...\\
    \star\Sigma &= ...\\
  \end{align}
  The AUC and its derivative at the starred parameters are
  \begin{align}
    \auc(\F,\G,\star{\beta}) &=...\\
    \auc'(\F,\G,\star{\beta}) &= ...
  \end{align}
  The derivative is $0$ iff ((eigenvector condition)). In terms of the
  normal means and variances, this condition is ((...)). 2 examples:
  1. independent data, ((...)), 2. proportional covariance
  matrices. The first is already implied by the general exponential
  family result [[ref above]] but not the second as the observations
  are not independent.

  show that derivative term can be unbounded. % must consider also
  % variance of infl function ie inverse fisher information, i.e., the
  % components of  % It is possible for $\auc'(\F,\G,\star{\beta})$ to have large
  % % components which are countered by variance of the parameter
  % % estimates. 
  When $\Sigma\approx\star\sigma$ [[not defined yet
  $\Sigma=\Sigma_0+\Sigma_1$]], the the influence function is
  approximately $O(|\Sigma|^{-1/2})$, the root of the inverse Fisher
  information, and $ \auc'(\F,\G,\star{\beta})$ is approximately
  $O(|\Sigma|^{1/2})$, so that the product, giving the entire
  non-Delong term, is approximately $O(1)$. Nevertheless it is
  possible to push the proportion so that the entire non-Delong term
  $(\star{\Sigma})^{-1} \auc'(\F,\G,\star{\beta})$ has large
  components.

  Let
  \begin{align}
    \Sigma_0 = ...
  \end{align}
  Then $(\star{\Sigma})^{-1} \auc'(\F,\G,\star{\beta}) \to \pm \infty \mathbbm{1}$ ((write out?)) as $\pi_0 \to 1$ and $\epsilon\to 0$ simultaneously. One therefore expects that under this scenario inference based on the Delong estimator will be faulty, as verified by simulation in ((ref simulation section)). ((possible to give result so that the sign of the derivative term is controlled, so that delong can be forced either to low fpr or low power?))
  
\end{example}

\section{Simulation}


\section{Discussion}
approach extends straightforwardly to other differentiable functions
of the data $f(x)$, not just the linear combination.

extends to discrete covariates (modified auc kernel)

\end{document}

