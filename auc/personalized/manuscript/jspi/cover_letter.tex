
\documentclass[10pt]{letter}

\usepackage{geometry}


\geometry{
	paper=a4paper, % Change to letterpaper for US letter
	top=3cm, % Top margin
	bottom=1.5cm, % Bottom margin
	left=4.5cm, % Left margin
	right=4.5cm, % Right margin
}

\usepackage[T1]{fontenc}
\usepackage[utf8]{inputenc}

\usepackage{stix}

\usepackage{microtype}

\signature{Haben Michael}

\address{Haben Michael \\Department of Mathematics \& Statistics\\ University of Massachusetts \\ 710 N. Pleasant St.\\ Amherst, MA 01003} % Your address and phone number

%----------------------------------------------------------------------------------------

\begin{document}

%----------------------------------------------------------------------------------------
%	ADDRESSEE SECTION
%----------------------------------------------------------------------------------------

\begin{letter}{Editorial Board, \emph{Journal of Statistical Planning and Inference} } % Name/title of the addressee

%----------------------------------------------------------------------------------------
%	LETTER CONTENT SECTION
%----------------------------------------------------------------------------------------

\opening{Dear Editors,}

Please find enclosed a copy of our manuscript, ``The Population and
Personalized AUCs,'' which we submit for
consideration for publication in the \emph{Journal of Multivariate
  Analysis}. The manuscript describes ways of extending the area under the curve, a popular measure of discrimination, to
accommodate clustered, possibly paired data. The need for such an
extension arises frequently, including in the authors' previous work with medical professionals,
yet to our knowledge statistical models and parameters have not been
carefully described. We describe two population parameters that may
produce substantially different inferences and conclusions. A careful
consideration of the assumptions and goals of analyzing clustered data
is therefore necessary, rather than the more ad hoc procedures
currently in employ. % This work complements and extends the
% foundational paper of N. Obuchowski on clustered AUC data that appeared in this journal in
% 1997.

We are re-submitting this manuscript (JSPI-D-23-00108) in response to suggestions from the executive editors:

\begin{enumerate}
\item \emph{Your paper in its present form cannot be assigned to an
    Associate editor. To start with, you need to define AUC in the
    title and the abstract since this abbreviation may not be familiar
    to the journal readership.}  We agree that, coming from a
  biostatistics readership where the abbreviation shows up in many
  article titles, we may have overlooked its opacity. We have written
  out the abbreviation and given a brief description of the comment in
  the abstract. An alternative that may be more familiar to JSPI's
  readership would be to use ``Mann-Whitney statistic'' rather than
  ``AUC'', though we believe the latter is currently much more popular
  in biostatistics and data science.


\item \emph{Second, the references on the subject are very
    limited. You need to add references and to carry out an exhaustive
    literature review.} We agree with this suggestion and have added a
  literature review. Our references have more than doubled. We have
  also included the 3 references cited in the data analysis.

\item \emph{Finally, when you submit the new version of the paper,
    make sure that the body of the paper precedes the appendix. After
    these revisions are done, your paper will be re-evaluated.}  We
  believe this must be a quirk of the uploading system. We will check
  the proofs to make sure parts have been ordered correctly.
\end{enumerate}
\vspace{2\parskip} % Extra whitespace for aesthetics
\closing{Sincerely,}


\end{letter}

\end{document}
