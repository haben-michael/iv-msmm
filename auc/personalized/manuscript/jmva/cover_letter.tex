
\documentclass[10pt]{letter}

\usepackage{geometry}


\geometry{
	paper=a4paper, % Change to letterpaper for US letter
	top=3cm, % Top margin
	bottom=1.5cm, % Bottom margin
	left=4.5cm, % Left margin
	right=4.5cm, % Right margin
}

\usepackage[T1]{fontenc}
\usepackage[utf8]{inputenc}

\usepackage{stix}

\usepackage{microtype}

\signature{Haben Michael}

\address{Haben Michael \\Department of Mathematics \& Statistics\\ University of Massachusetts \\ 710 N. Pleasant St.\\ Amherst, MA 01003} % Your address and phone number

%----------------------------------------------------------------------------------------

\begin{document}

%----------------------------------------------------------------------------------------
%	ADDRESSEE SECTION
%----------------------------------------------------------------------------------------

\begin{letter}{Editorial Board, \emph{Journal of Multivariate Analysis} } % Name/title of the addressee

%----------------------------------------------------------------------------------------
%	LETTER CONTENT SECTION
%----------------------------------------------------------------------------------------

\opening{Dear Editors,}

Please find enclosed a copy of our manuscript, ``The Population and
Personalized AUCs,'' which Professor Tian and I submit for
consideration for publication in the \emph{Journal of Multivariate
  Analysis}. This manuscript was recently rejected as too long
(Manuscript No. JMVA-D-23-00061) and we have shortened it
considerably, now standing at 22 pages total (15 pages + proofs). (The decision letter also mentioned to ``motivate that although your work is about bivariate variables it should be considered to be multivariate,'' which we take to be general advice  that all submissions to this journal should meet, since nothing in this manuscript discusses bivariate data.)

The manuscript describes ways of extending the area under the curve to
accommodate clustered, possibly paired data. The need for such an
extension arises frequently, including in the authors' previous work with medical professionals,
yet to our knowledge statistical models and parameters have not been
carefully described. We describe two population parameters that may
produce substantially different inferences and conclusions. A careful
consideration of the assumptions and goals of analyzing clustered data
is therefore necessary, rather than the more ad hoc procedures
currently in employ. % This work complements and extends the
% foundational paper of N. Obuchowski on clustered AUC data that appeared in this journal in
% 1997.

We have not submitted the manuscript elsewhere, and look forward to your review and
comments.

\vspace{2\parskip} % Extra whitespace for aesthetics
\closing{Sincerely,}


\end{letter}

\end{document}
