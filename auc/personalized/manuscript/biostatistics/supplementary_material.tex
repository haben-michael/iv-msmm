\documentclass[12pt]{article}
\usepackage{amsthm}
\usepackage{bbm}
\usepackage{amssymb}
\usepackage{mathtools}
\mathtoolsset{showonlyrefs,showmanualtags}
\usepackage{etoolbox}
% \usepackage{booktabs}
% \usepackage{url}
\usepackage{setspace}
\usepackage[margin=1in]{geometry}
\usepackage{authblk}
\usepackage{natbib}
% \usepackage[page]{appendix}
\usepackage[nomarkers,nolists]{endfloat}
\usepackage{graphicx}
% \usepackage{caption}
% \usepackage{subcaption}
\usepackage{tikz}
\usetikzlibrary{calc}
\usepackage{subfig}
\usepackage{enumitem}
\usepackage{array} % tables with fixed width and alignment
\usepackage{chngpage} % for the too long table at the end
\usepackage{xr} % reference proofs in suppl material
\DeclareMathOperator{\AUC}{AUC}
\DeclareMathOperator{\V}{Var}
\DeclareMathOperator{\cov}{Cov}
\DeclareMathOperator{\corr}{Corr}
\DeclareMathOperator{\sd}{sd}
% \renewcommand{\t}[1]{\tilde{#1}}
\newcommand{\h}[1]{\hat{#1}}
\newcommand{\I}{I}
% \newcommand{\partiall}[1]{\frac{\partial}{\partial #1}}
% \newcommand{\gm}{\theta}
\newcommand{\E}{E}
\renewcommand{\P}{P}
% \newcommand{\mean}[1]{\overline{#1}}
% \newcommand{\sel}[1]{#1^*}
% \newcommand{\biasratio}{b}% {$(E|S_1-S_2|)^2/\E(S^2)$}
\newcommand{\cind}{\perp \!\!\! \perp}
\newcommand{\aucindiv}{\theta_{11}}%{\AUC}
\newcommand{\aucpop}{\theta_{12}}%{\AUC_{\cind}}
\newcommand{\aucindivhat}{\hat{\theta}_{11}}%{\AUC}
\newcommand{\aucpophat}{\hat{\theta}_{12}}%{\AUC_{\cind}}
\newcommand{\kernel}{\psi}
\newcommand{\Kernel}{\psi}
\newcommand{\B}{B}
% \newcommand{\W}[1]{X_{#1},Y_{#1}}
\newcommand{\W}[1]{W_{#1}}
\newcommand{\bnd}{a}
% \renewcommand{\V}{c_{00}}
\newcommand{\seqspace}{V}%{c_{00}}
\renewcommand{\d}{\phi}
\newcommand{\Pind}{P_{\cind}}
\newcommand{\A}[1]{P(A^{(n)}_{#1})}
\newtheorem{theorem}{Theorem}
\newtheorem{proposition}[theorem]{Proposition}
\newtheorem{lemma}[theorem]{Lemma}
\newtheorem{corollary}[theorem]{Corollary}
\makeatletter
\def\input@path{{../input/}{../figs/}}
\graphicspath{{../figs/}}
\makeatother
\newcommand{\PreserveBackslash}[1]{\let\temp=\\#1\let\\=\temp}
\newcolumntype{C}[1]{>{\PreserveBackslash\centering}p{#1}}
\newtoggle{commenttoggle}
\togglefalse{commenttoggle}
\newcommand{\comment}[1]{
  \iftoggle{commenttoggle}{
    {\normalsize{\color{red}{ #1}}\normalsize}
  }
  {}
}
\externaldocument{manuscript}
\doublespacing
\title{Supplementary Material for ``The Population and Personalized AUCs''}
\author[1]{Haben Michael}
\author[2]{Lu Tian}
\affil[1]{University of Massachusetts}
\affil[2]{Stanford University}
\date{}

\begin{document}

% \maketitle
% \noindent\textsc{Abstract:} We consider two generalizations of the AUC to accommodate
% clustered data. We describe situations in which the two cluster AUCs
% diverge and other situations in which they coincide. We apply the
% results to data collected on urban policing behavior.\\
% \textsc{Keywords:} AUC, Confounding, Clustered data, Simpson's paradox



  \begin{proof}[Proof of Proposition \ref{proposition:aucpop}]
    \begin{enumerate}
    \item By the LLN $\I^2/MN \to 1/(\E (M)\E (N))$ almost surely and by Lemma \ref{corollary:convergence} $\sum_{i,j}\Kernel_{ij}/\I^2 \to \E \Kernel_{12}$ almost surely. Conditioning on the sample,
      \begin{align}
        \E \kernel(\xi_\I,\eta_\I) &= \E( \E (\kernel(\xi_\I,\eta_\I) \mid (X_1,Y_1,M_1,N_1),\ldots,(X_\I,Y_\I,M_\I,N_\I))\\
                                   &= \E\left(\frac
                                     {\sum_{1\le i,j\le\I}\sum_{1\le k\le M_i,1\le l\le N_j}\kernel(X_{ik},Y_{jl})}
                                     {\sum_{i=1}^\I M_i \sum_{i=1}^\I N_i} \right)\\
                                   &= \E\left(\frac{\sum_{1\le i,j\le\I}\Kernel_{ij}}{\sum_{i=1}^\I M_i \sum_{i=1}^\I N_i} \right) \to \frac{\E\Kernel_{12}}{\E (M) \E (N)}=\aucpop.
      \end{align}
      The limit is justified since $\sum_{i,j}\psi_{i,j}/(\sum_i M_i\sum_i N_i)\le 1$. %dominated convergence, given the boundedness of $\I^2/MN$ and moment condition on $\Kernel$.

    \item The second part follows on showing that $(\xi_\I,\eta_\I)\to (\xi_\infty,\eta_\infty)$ setwise. % \comment{if arguing $\theta_{12}(P_\I)\to\theta_{12}(P_\infty)$ need to show $P_\I$ concentrates on a continuity set of $P_\infty$.}
      For $a,b\in\mathbb{R}$, by a similar argument as above,
      \begin{align}
        \P(\xi_\I<a,\eta_\I<b) &=\E\left(\frac
                                     {\sum_{1\le i,j\le\I}\sum_{1\le k\le M_i,1\le l\le N_j}\{X_{ik}<a,Y_{jl}<b\}}
                                 {\sum_{i=1}^\I M_i \sum_{i=1}^\I N_i} \right)\\
                               &\to \frac{\E\left(\sum_{k=1}^{M_1}\{X_{1k}<a\}\right)}{\E (M)}
                                 \frac{\E\left(\sum_{l=1}^{N_1}\{Y_{1l}<b\}\right)}{\E (N)}.
      \end{align}
      The probability of sampling an element from a cluster of size
      $M=m$ given an initial segment of $\I$ samples
      $(X_1,Y_1,M_1,N_1),\ldots,(X_\I,Y_\I,M_\I,N_\I),$ is
      $\frac{m\sum_{i=1}^\I\{M_i=m\}}{\sum_{i=1}^\I M_i}$. Along almost
      any sequence of samples as $\I\to\infty$ this relative frequency
      tends to $\frac{m\P(M=m)}{\E (M)}$. Therefore
      \begin{align}
        \P(\xi_\infty < a) &= \sum_{m=1}^\infty\P(\xi_\infty < a \mid \xi_\infty\text{ is sampled from a cluster of size }m)\cdot\\
        &\hspace{.5in}\P(\xi_\infty\text{ is sampled from a cluster of size }m)\\
                           &= \sum_{m=1}^\infty\frac{1}{m}\sum_{k=1}^m\P(X_{1k}<a\mid M=m)\frac{m\P(M=m)}{\E (M)}\\
                           &=\frac{1}{\E M}\sum_{m=1}^\infty\sum_{k=1}^m\P(X_{1k}<a\mid M=m)\P(M=m)\\
        &=\frac{1}{\E (M)}\E\left(\sum_{k=1}^M\{X_{1k}<a\}\right).
      \end{align}
      Analogously,
      $$
      \P(\eta_\infty < a)=\frac{1}{\E (N)}\E\left(\sum_{l=1}^N\{X_{1l}<a\}\right).
      $$
      The product is the limit of
      $\P(\xi_\I<a,\eta_\I<b)$ given above.
    \end{enumerate}
  \end{proof}
    The following lemma gives a convergence result for a two-sample $U$-statistic with kernel of degree $(1,1)$ where the data is paired. The corresponding definitions and result for independent samples is given in, e.g., \citet{lee2019}. Let $\seqspace$ denote the space of finite sequences.%\comment{Need to define $V$, the space where $X,Y$ lie. These are vectors of variable length so $V$ should be at least as big as $c_{00}$}

  %   \begin{lemma}\label{lemma:hajek} Given a sample $(X_0,Y_0),(X_1,Y_1),\ldots,(X_\I,Y_\I)$ on
  %   $\seqspace\times\seqspace$ IID according to $\P$ and a
  %   function $\Kernel: \seqspace\times\seqspace \to \mathbb{R}$ in $L^2(\P)$, define
  %   $$
  %   U_\I = \I^{-2}\sum_{1\le i,j\le\I}\Kernel(X_i,Y_j)
  %   $$
  %   and
  %   $$
  %   \hat{U}_\I = \I^{-1}\sum_{i=1}^\I\left(\E(\Kernel(X_i,Y_0)\mid X_i,Y_i) + \E(\Kernel(X_0,Y_i)\mid X_i,Y_i)\right) - 2\E\Kernel(X_1,Y_2).
  %   $$
  %   Then
  %   $$
  %   \E(U_\I-\E U_\I-\hat{U}_\I)^2=O(\I^{-2}).
  %   $$
  % \end{lemma}

  % \begin{proof}[Proof of Lemma \ref{lemma:hajek}]

  %   Let $V_\I=(\I)^{-1}_2\sum_{\stackrel{1\le i,j\le \I}{i\neq j}}\psi_{ij}.$ Then,

  %   \begin{align}
  %     \E(V_\I-\E\Kernel_{12})^2 &= (\I_2)^{-2}\E(\sum_{i\neq j}\Kernel_{ij})^2-\left(\E\Kernel_{12}\right)^2\\%...\V\left( (\I)_2^{-1}\sum_{i\neq j}\Kernel_{ij}\right) \\
  %                               &= \frac{\I-2}{\I(\I-1)}\left(\E\Kernel_{12}\Kernel_{13}+\E\Kernel_{12}\Kernel_{13}+2\E\Kernel_{12}\Kernel_{31}\right) + \left(\frac{(\I)_4}{(\I)^2_2}-1\right)\left(\E\Kernel_{12}\right)^2+O(\I^{-2})\\
  %                               &= \frac{\I-2}{\I(\I-1)}\left(\I\E\hat{U}_\I^2+4\left(\E\Kernel_{12}\right)^2 \right) + \left(\frac{(\I)_4}{(\I)^2_2}-1\right)\left(\E\Kernel_{12}\right)^2+O(\I^{-2})\\
  %                                                           % \\&=...\V(\E(\Kernel(X_1,Y_0)\mid X_1,Y_1)+\E(\Kernel(X_0,Y_1)\mid X_1,Y_1))\\
  %                                                           &=\frac{\I-2}{\I-1}\V(\hat{U}_\I) + O(\I^{-2}).
  %   \end{align}
  %   By a similar computation, or by viewing $\hat{U}_\I$ as the H\'ajek projection of $V_\I$ \cite{lee2019},
  %   \begin{align}
  %     \E\left((V_\I-\E\Kernel_{12})\hat{U}_\I\right)=\E\hat{U}_\I^2.
  %   \end{align}
  %   % \begin{align}
  %   %   &\cov(\Kernel_{12},\E(\Kernel(X_1,Y_0)\mid X_1,Y_1)+\E(\Kernel(X_0,Y_1)\mid X_1,Y_1)+\E(\Kernel(X_2,Y_0)\mid X_2,Y_2)+\E(\Kernel(X_0,Y_2)\mid X_2,Y_2))\\
  %   %   &\qquad = ...\\
  %   %   &\qquad = \V(\E(\Kernel(X_1,Y_0)\mid X_1,Y_1)+\E(\Kernel(X_0,Y_1)\mid X_1,Y_1))\\
  %   %   &\qquad=\I\V(\hat{U}_\I).
  %   % \end{align}
  %   The variance of the IID sum $\hat{U}_\I$ is $O(\I^{-1})$. Therefore,
  %   \begin{align}
  %     \E\left(U_\I-\E U-\hat{U}_\I\right)^2 &= \E\left(V_\I-\E\Kernel_{12}-\hat{U}_\I\right)^2 +  O(\I^{-2})\\
  %                                           &= \left(\frac{\I-2}{\I-1}-2+1\right)\E\hat{U}_\I^2 + O(\I^{-2})\\
  %                                           &= O(\I^{-2}).
  %   \end{align}
  % \end{proof}

    \begin{lemma}\label{lemma:hajek} Given a sample $(X_0,Y_0),(X_1,Y_1),\ldots,(X_\I,Y_\I)$ on
    $\seqspace\times\seqspace$ IID according to $\P$ and a
    function $\Kernel: \seqspace\times\seqspace \to \mathbb{R}$ in $L^2(\P)$, define
    $$
    U_\I = \I^{-2}\sum_{\substack{1\le i,j\le\I\\i\neq j}}\Kernel(X_i,Y_j),\qquad
    V_\I = \I^{-2}\sum_{1\le i,j\le\I}\Kernel(X_i,Y_j),
    $$
    and
    $$
    \hat{U}_\I = \I^{-1}\sum_{i=1}^\I\left(\E(\Kernel(X_i,Y_0)\mid X_i,Y_i) + \E(\Kernel(X_0,Y_i)\mid X_i,Y_i)\right) - 2\E\Kernel(X_1,Y_2).
    $$
    Then
    $$
    \E(U_\I-\E U_\I-\hat{U}_\I)^2=O(\I^{-2})\text{ and }  \E(V_\I-\E V_\I-\hat{U}_\I)^2=O(\I^{-2}).
    $$
  \end{lemma}

  \begin{proof}[Proof of Lemma \ref{lemma:hajek}]

    Define
    $$
    \overline\Kernel_{ij}=\Kernel(X_i,Y_j) - \E(\Kernel(X_i,Y_0)\mid X_i,Y_i) - \E(\Kernel(X_0,Y_j)\mid X_j,Y_j) + \E\Kernel(X_0,Y_0).
    $$
    Then, for $i\neq j$, $\E (\overline\Kernel_{ij}\mid (X_i,Y_i))=\E (\overline\Kernel_{ij}\mid (X_j,Y_j))=0$, implying
    \begin{align}
      \E(U_\I-\E U_\I - \hat{U}_\I)^2 &= \E\left((\I)^{-1}_2\sum_{i\neq j} \overline\Kernel_{ij}\right)\\
                                      &= (\I)_2^{-2}\sum_{i\neq j}\E \overline\Kernel_{ij}^2 + O(\I^{-2})\\
                                      &=O(\I^{-2}).
    \end{align}

    For the second equation,
    \begin{align}
      \E(U_\I-\E U_\I - V_\I+\E V_\I)^2 &=  \I^{-2}\E\left((\I)_2^{-1}\sum_{i\neq j}\Kernel_{ij} - \E \Kernel_{11}+\E\Kernel_{12}\right)^2\\
                                        &\le \I^{-2}\left((\I)_2^{-1}\sum_{i\neq j}\E(\Kernel_{ij}-\E\Kernel_{11}+\E\Kernel_{12})^2\right)\\
      &=O(\I^{-2}).
    \end{align}

  \end{proof}

  \begin{corollary}\label{corollary:convergence}

    With the same setup as Lemma \ref{lemma:hajek}, $U_\I-\E U_\I\to 0$ a.s. and $\sqrt{\I}(U_\I-\E U_\I)/\sqrt{\V(U_\I)}\to\mathcal{N}(0,1)$ in distribution.%\comment{$\hat U_\I\to\E U$ a.s. and $\sqrt{\I}(\hat U_\I-\E U)/\sqrt{\V(U_\I)}\to\mathcal{N}(0,1)$ in distribution.}
  \end{corollary}
  \begin{proof}[Proof of Corollary \ref{corollary:convergence}]
    By Lemma \ref{lemma:hajek}, $U_\I-\E U_\I\to\hat{U}_\I$ a.s. and $\sqrt{\I}(U_\I- \E U_\I-\hat{U}_\I)\to 0$ in quadratic mean, and $\hat{U}_\I$ is an IID sum subject to the usual LLN and CLT.
  \end{proof}

\begin{proof}[Proof of Proposition \ref{proposition:reduction}]
  \begin{align}
    \aucindiv(\P) &= \E\left(\frac{\sum_{k=1}^M\sum_{l=1}^N\kernel(X_{1k},Y_{1l})}{MN}\right)\\
                  &=\E\left(\frac{1}{MN}\E\left(\sum_{k=1}^M\sum_{l=1}^N\kernel(X_{1k},Y_{1l}) \mid M,N\right)\right)\\
                  &=\E\left(\frac{1}{MN}MN\E(\kernel(X_{11},Y_{11}\mid M,N))\right) = \E\kernel(X_{11},Y_{11}).
  \end{align}
  % Lemma \ref{lemma:conditional wald} was used to get the third equality.

  % If $\E\kernel(X_{1k},Y_{1l})$ does not depend on $k,l$, then neither does $\E\kernel(X_{1k},Y_{2l})$.
  Similar to the above,
  \begin{align}
    \aucpop(\P) &= \frac{\E\left(\sum_{k=1}^{M_1}\sum_{l=1}^{N_2}\kernel(X_{1k},Y_{2l})\right)}{\E(M)\E(N)}\\
    &=\frac{\E(M)\E(N)\E\kernel(X_{11},Y_{21})}{\E(M)\E(N)} = \E\kernel(X_{11},Y_{21}).
  \end{align}
\end{proof}


\begin{lemma}\label{lemma:conditional wald}
  Given integrable random variables $M,V,X_1,X_2,\ldots,$ such that $M\in\{1,2,\ldots\}$ and $\sum_{i=1}^\infty E(|X_i|;M\ge i)<\infty$,
  \begin{align}
    \E\left(\sum_{i=1}^M X_i \bigg\vert M,V\right) = \sum_{i=1}^M \E(X_i\mid M,V)
  \end{align}
\end{lemma}
\begin{proof}[Proof of Lemma \ref{lemma:conditional wald}]
  \begin{align}
    \E\left(\sum_{i=1}^M X_i\bigg\vert M,V\right)
    &=  \E\left(\sum_{m=1}^\infty\{M=m\}\sum_{i=1}^m X_i\bigg\vert M,V\right)\\
    &= \sum_{m=1}^\infty \E\left(\{M=m\}\sum_{i=1}^m X_i\bigg\vert M,V\right)\\
    &=\sum_{m=1}^\infty \sum_{i=1}^m\{M=m\}\E(X_i\mid M,V)\\
    &=\sum_{i=1}^M\E(X_i\mid M,V),
    % &=\sum_{i=1}^\infty\{M\ge i\}\E(X_i\mid M,V)
  \end{align}
the interchange in the second equality allowed since $E\left|\sum_{i=1}^MX_i\right| \le \sum_{i=1}^\infty E(|X_i|;M\ge i)<\infty.$
\end{proof}


\begin{proof}[Proof of Lemma \ref{lemma:bounds}]
  Define for $n\in\mathbbm{N}$ approximations to $\aucindiv$ and $\aucpop$ by
  \begin{align}
    A_{ij}^{(n)} &= \left\{(x,y) : \frac{i}{2^n}\le x<\frac{i+1}{2^n},
    \frac{j}{2^n}\le y<\frac{j+1}{2^n}\right\},
    \hspace{.1in}-2^{2n}\le i,j < 2^{2n}-1\\
    \aucindiv^{(n)} &= \sum_{i=-2^{2n}}^{2^{2n}-1}\sum_{j=i+1}^{2^{2n}-1} \P(A_{ij}^{(n)})
                      + \frac{1}{2}\sum_{i=-2^{2n}}^{2^{2n}-1} \P(A_{ii}^{(n)})\\
    \aucpop^{(n)} &= \sum_{i=-2^{2n}}^{2^{2n}-1}\sum_{j=i+1}^{2^{2n}-1} \Pind(A_{ij}^{(n)})
                    + \frac{1}{2}\sum_{i=-2^{2n}}^{2^{2n}-1} \Pind(A_{ii}^{(n)}).
  \end{align}

  % and analogously for an approximation $\aucpop^{(n)}$ to $\aucpop$ using the product of the marginals $\Pind$ rather than $\P$
%   \begin{align}
% ,
%   \end{align}
  Since $\bigcup_n \bigcup_i\bigcup_{j>i} A_{ij}^{(n)} = \{x < y\}$ and
    $\bigcap_n\bigcup_i A_{ii}^{(n)} = \{x=y\},$
  % \begin{align}
    % \bigcup_n \bigcup_i\bigcup_j A_{ij}^{(n)} &= \{x < y\}\\
    % \bigcap_n\bigcup_i A_ii^{(n)} &= \{x=y\}
  % \end{align}
    by continuity of measure $\aucindiv^{(n)}\to\aucindiv$ and $\aucpop^{(n)}\to\aucpop$. Therefore, it is enough to establish the inequality \eqref{eqn:lemma:bounds:conclusion} for $\aucindiv^{(n)}$ and $\aucpop^{(n)}$.

    Fixing $n$,
    \begin{align}
      &\sum_{i=-2^{2n}}^{2^{2n}-2}\sum_{j=i+1}^{2^{2n}-1} \Pind(A_{ij}^{(n)})
      = \sum_{i=-2^{2n}}^{2^{2n}-2}\sum_{j=i+1}^{2^{2n}-1} \Pind(A_{ij}^{(n)})\\
      &= \sum_{i=-2^{2n}}^{2^{2n}-2}\sum_{j=i+1}^{2^{2n}-1} \Pind(\frac{i}{2^n}\le x<\frac{i+1}{2^n})
        \Pind(\frac{j}{2^n}\le y<\frac{j+1}{2^n})\\
      &\ge \sum_{i=-2^{2n}}^{2^{2n}-2}\sum_{j=i+1}^{2^{2n}-1}
        (\A{ii}+\sum_{k=i+1}^{2^{2n}-1}\A{ik})
        (\A{jj}+\sum_{l=-2^{2n}}^{j-1}\A{lj})\\
      &= \sum_{i=-2^{2n}}^{2^{2n}-2}\sum_{j=i+1}^{2^{2n}-1}\left(
        \sum_{k=i+1}^{2^{2n}-1}\A{ik}\sum_{l=-2^{2n}}^{j-1}\A{lj} +
        \A{ii}\sum_{l=-2^{2n}}^{j-1}\A{lj} \right.\\
        & \left. +
      \A{jj}\sum_{k=i+1}^{2^{2n}-1}\A{ik} +
        \A{ii}\A{jj}
    \right).
    \end{align}

    We lower bound the first three terms in parentheses.

    First term:
    \begin{align}
     &      \sum_{i=-2^{2n}}^{2^{2n}-2}\sum_{j=i+1}^{2^{2n}-1}
      \sum_{k=i+1}^{2^{2n}-1}\A{ik}\sum_{l=-2^{2n}}^{j-1}\A{lj}\\
      &=            \sum_{i=-2^{2n}}^{2^{2n}-2}\sum_{k=i+1}^{2^{2n}-1}\A{ik}
        \sum_{j=i+1}^{2^{2n}-1}
      \sum_{l=-2^{2n}}^{j-1}\A{lj}\\
      &\ge            \sum_{i=-2^{2n}}^{2^{2n}-2}\sum_{k=i+1}^{2^{2n}-1}\A{ik}
        \sum_{j=i+1}^{2^{2n}-1}
      \sum_{l=i}^{j-1}\A{lj}\\
      &=            \sum_{i=-2^{2n}}^{2^{2n}-2}\sum_{k=i+1}^{2^{2n}-1}\A{ik}
        \sum_{l=i}^{2^{2n}-2} \sum_{j=l+1}^{2^{2n}-1}   \A{lj}\\
      &=            \sum_{i=-2^{2n}}^{2^{2n}-2}\sum_{k=i+1}^{2^{2n}-1}\A{ik}
         \sum_{j=i+1}^{2^{2n}-1} \A{ij} +
        \sum_{i=-2^{2n}}^{2^{2n}-2}\sum_{k=i+1}^{2^{2n}-1}\A{ik}
        \sum_{l=i+1}^{2^{2n}-2} \sum_{j=l+1}^{2^{2n}-1} \A{lj}\\
      &\ge \sum_{i=-2^{2n}}^{2^{2n}-2}\sum_{j=i+1}^{2^{2n}-1}\A{ij}^2 +
        \sum_{i=-2^{2n}}^{2^{2n}-2}\sum_{k=i+1}^{2^{2n}-2}
        \sum_{j=k+1}^{2^{2n}-1} \A{ij} \A{ik} +
                \sum_{i=-2^{2n}}^{2^{2n}-2}\sum_{k=i+1}^{2^{2n}-1}\A{ik}
        \sum_{l=i+1}^{2^{2n}-2} \sum_{j=l+1}^{2^{2n}-1} \A{lj}\\
      &= \underset{\substack{i\neq k \text{ or } j\neq l\\j>i\text{ and }l>k}}
      {\sum\sum\sum\sum}\A{ij}\A{kl} + \sum_{i=-2^{2n}}^{2^{2n}-2}\sum_{j=i+1}^{2^{2n}-1}\A{ij}^2 \\
      &= \frac{1}{2}\left(\sum_{i=-2^{2n}}^{2^{2n}-2}\sum_{j=i+1}^{2^{2n}-1}\A{ij}\right)^2 +
         \frac{1}{2}\sum_{i=-2^{2n}}^{2^{2n}-2}\sum_{j=i+1}^{2^{2n}-1}\A{ij}^2 .
    \end{align}


    Middle two terms:

    \begin{align}
            & \sum_{i=-2^{2n}}^{2^{2n}-2}\sum_{j=i+1}^{2^{2n}-1}\left(
        \A{ii}\sum_{l=-2^{2n}}^{j-1}\A{lj} +
              \A{jj}\sum_{k=i+1}^{2^{2n}-1}\A{ik} \right)\\
      &= \sum_{i=-2^{2n}}^{2^{2n}-2}\A{ii}\sum_{l=i}^{2^{2n}-2}
        \sum_{j=l+1}^{2^{2n}-1}\A{lj} +
        \sum_{j=-2^{2n}+1}^{2^{2n}-1}\A{jj}\sum_{i=-2^{2n}}^{j-1}
        \sum_{k=i+1}^{2^{2n}-1}\A{ik}\\
      &= \sum_{i=-2^{2n}}^{2^{2n}-2}\A{ii}\sum_{l=i}^{2^{2n}-2}
        \sum_{j=l+1}^{2^{2n}-1}\A{lj} +
        \sum_{i=-2^{2n}+1}^{2^{2n}-1}\A{ii}\sum_{l=-2^{2n}}^{i-1}
        \sum_{j=l+1}^{2^{2n}-1}\A{lj}\\
            &=\left(\sum_{i=-2^{2n}}^{2^{2n}-1}\A{ii}\right)
              \left(\sum_{l=-2^{2n}}^{2^{2n}-2}\sum_{j=l+1}^{2^{2n}-1}\A{lj}\right).
    \end{align}
    The second-to-last equality is just renaming indices.

    % The final term is

    % \begin{align}
    %   & \sum_{i=-2^{2n}}^{2^{2n}-2}\sum_{j=i+1}^{2^{2n}-1} \A{ii}\A{jj}
    %   =
          %       \end{align}

    With these lower bounds,
    \begin{align}
    \aucpop^{(n)} &= \sum_{i=-2^{2n}}^{2^{2n}-1}\sum_{j=i+1}^{2^{2n}-1} \Pind(A_{ij}^{(n)})
                    + \frac{1}{2}\sum_{i=-2^{2n}}^{2^{2n}-1} \Pind(A_{ii}^{(n)})\\
                  &\ge \frac{1}{2}\left(\sum_{i=-2^{2n}}^{2^{2n}-2}\sum_{j=i+1}^{2^{2n}-1}\A{ij}\right)^2 +
                    \left(\sum_{i=-2^{2n}}^{2^{2n}-1}\A{ii}\right)
                    \left(\sum_{l=-2^{2n}}^{2^{2n}-2}\sum_{j=l+1}^{2^{2n}-1}\A{lj}\right) +\\
                  &\sum_{i=-2^{2n}}^{2^{2n}-2}\sum_{j=i+1}^{2^{2n}-1} \A{ii}\A{jj}
                    + \frac{1}{2}\sum_{i=-2^{2n}}^{2^{2n}-1} \P(A_{ii}^{(n)})^2\\
                  &= \frac{1}{2}\left(\sum_{i=-2^{2n}}^{2^{2n}-2}\sum_{j=i+1}^{2^{2n}-1}\A{ij} +
                    \sum_{i=-2^{2n}}^{2^{2n}-1} \P(A_{ii}^{(n)})\right)^2\\
      &= \frac{1}{2}\left(\aucindiv^{(n)} +\frac{1}{2}\sum_{i=-2^{2n}}^{2^{2n}-1} \P(A_{ii}^{(n)})\right)^2.\\
      &= \frac{1}{2}\left(\aucindiv^{(n)} +\frac{1}{2}\P(X=Y)\right)^2+o(1).\\
    \end{align}
     The upper bound then follows by the same symmetry argument as given in Section \ref{section:simplifications}.

  \end{proof}

\begin{proof}[Proof of Theorem \ref{theorem:bounds}]

  % ($m\ge 1, n\ge 1$)
  With
  $$
  \aucindiv = \frac{1}{mn}\E(\Kernel_{11}) = \frac{1}{mn}\sum_{i,j}(\P(X_{1i}<Y_{1j})+\frac{1}{2}\P(X_{1i}=Y_{1j}))
  $$
  Lemma \ref{lemma:bounds} gives
  \begin{align}
    \aucpop &= \frac{1}{mn}\E(\Kernel_{12}) = \frac{1}{mn}\sum_{i,j}(P(X_{1i}<Y_{2j})+\frac{1}{2}\P(X_{1i}=Y_{2j}))\\
    &\ge \frac{1}{mn}\sum_{i,j}\frac{1}{2}(\P(X_{1i}<Y_{1j})+\P(X_{1i}=Y_{1j}))^2\\
  % \end{align}
  % continuing with Jensen's inequality
  % \begin{align}
    &\ge \frac{1}{2}\left(\frac{1}{mn}\sum_{i,j}(\P(X_{1i}<Y_{1j})+\P(X_{1i}=Y_{1j}))\right)^2\\
    &= \frac{1}{2}\left(\aucindiv + \frac{1}{2mn}\sum_{i,j}\P(X_{1i}=Y_{1j})\right)^2.
  \end{align}
  The second inequality is Jensen's inequality, which is tight when
  the pairwise AUCs are all equal.  The other bound follows similarly.
  % [[maybe switch to $P$ and $\P_\cind$ notation above]]
\end{proof}

  \begin{proof}[Proof of Theorem \ref{theorem:asymptotic}]

    By Lemma \ref{lemma:hajek},%Corollary \ref{corollary:convergence}
  \begin{align}
    \sqrt{\I}\left(
    \frac{(\I)_2^{-1}\sum_{i\neq j}\Kernel_{ij}-\E\Kernel_{12}}{\sd(\sqrt{\I}(\I)_2^{-1}\sum_{i\neq j}\Kernel_{ij})}
    ,
    \frac{\I^{-2}\sum_{i,j}M_iN_j - \E (M)\E (N)}
    {\sd(\I^{-3/2}\sum_{i, j}M_iN_j ) }
    ,
     \frac{\I^{-1}\sum_i \Kernel_{ii}/(M_iN_i) - \E(\Kernel_{11}/M_1N_1)}
     {\sd(\Kernel_{11}/M_1N_1)}
    \right)
  \end{align}
  converges to
  \begin{align}
    \I^{-1/2}\sum_{i=1}^\I & \left(
     \frac{\E(\Kernel_{i0}\mid\W{i})+\E(\Kernel_{0i}\mid\W{i})-2\E\Kernel_{12}}
    {\sd(\E(\Kernel_{10}\mid\W{1})+\E(\Kernel_{01}\mid\W{1}))}
    ,
     \frac{M_i\E (N) + N_i\E (M) - 2\E (M)\E (N)}
     {\sd(M_1\E (N) + N_1\E (M))}
     ,\right.\\
     &\left.
     \frac{ \Kernel_{ii}/(M_iN_i) - \E(\Kernel_{11}/M_1N_1)}
    {\sd(\Kernel_{11}/M_1N_1)}
    \right)
  \end{align}
  in mean-square. The latter is an IID sum with finite covariance matrix and is asymptotically normal by the usual CLT.
  % \begin{align}
  % \end{align}
  Applying the delta method with the function
  $(x,y,z)\mapsto (x/y,z)$, with derivative
  $$
\left.  \begin{pmatrix}
    1/y & -x/y^2 & 0 \\
    0 & 0 & 1
  \end{pmatrix} \right|_{(x,y)=(\aucpop,\E (M) \E (N))}
  $$
  for $y\neq 0$, i.e., $E(M)\neq 0, E(N)\neq 0$, gives the asymptotic normality of $(\aucindiv,\aucpop)$.
  The asymptotic covariance matrix is given by delta method. %\comment{track down the 1/2 term--see notes P.3}

%   \begin{align}
%   \end{align}
%   and by continuity of the delta method function, the same must hold of [[ref target vector]].


%   first two components are nearly two sample u-statistics (but dependence across same cluster), last being an average is a 1-sample u-statistic, so we use similar methods.


%   Two steps: Show the asymptotic normality of
%   \begin{align}
%    \sqrt{\I}((I)^{-1}_2\sum_{i,j}\Kernel_{ij} - \E\Kernel_{12}, (\I)^{-2}\sum_{i,j}M_iN_j - \E(MN), \I^{-1}\sum_i \Kernel_{ii}/(M_iN_i)).
%   \end{align}
%   Then apply delta method with the function $(x,y,z)\mapsto (x/y,z)$, with derivative
%   $$
%   \begin{pmatrix}
%     1/y & -x/y^2 & 0 \\
%     0 & 0 & 1
%   \end{pmatrix}
%   $$
%   for $y\neq 0$, i.e., $E(M)\neq 0, E(N)\neq 0$.


%   For first step: Whow asy normality of [[ref above]] by showing each component
%   converges in $L_2$ to an iid sum. Third component is already an iid sum, so only need to take care of the first two components.

% condition on setof sums ....
%   falling factorial notation $(\I)_n=\prod_{i=0}^{n-1}(I-i)$ for $n\ge 1$.

%   first component:

%   The $L_2$ projection of $\E((\I)_2^{-1}\sum_{i\neq j}\Kernel_{ij} - \E\Kernel_{12}$ onto the subspace ... is
%   \begin{align}
%     \I^{-1}\sum_{i=1}^\I(\E(\Kernel_{i0}\mid \W_i) + \E(\Kernel_{0i}\mid\W_i)) -2\E\Kernel_{12}
%   \end{align}
%   and the variance of the projection, multiplied by $\I$, is
%   \begin{align}
%     \V(\E(\Kernel_{12}\mid\W_1) + \E(\Kernel_{21}\mid \W_1)).
%   \end{align}
%   The normalized variance of the first component of [[ref vector]] is
%   \begin{align}
%     \I\V((\I)_2^{-1}\sum_{i\neq j}\Kernel_{ij}) &=
%                                                   \I(\I_2^{-2}(\I)_3(\cov(\Kernel_{12},\Kernel_{13})+\cov(\Kernel_{21},\Kernel_{31}) +2\cov(\Kernel_{12},\Kernel_{31}))\\
%                                                 &= \frac{\I-2}{\I-1}(\E\Kernel_{12}\Kernel_{13}+\E\Kernel_{21}\Kernel_{31}+2\E\Kernel_{12}\Kernel_{31}-4(\E\Kernel_{12})^2)\\
%                                                 &=\frac{\I-2}{\I-1}(\E(\E(\Kernel_{12}\mid\W_1)^2)+\E(\E(\Kernel_{21}\mid\W_1)^2) + 2\E(\E(\Kernel_{12}\mid\W_1)\E(\Kernel_{21}\mid\W_1)) - 4(\E\Kernel_{12})^2)\\
%                                                 &=\frac{\I-2}{\I-1}(\E(\E(\Kernel_{12}\mid\W_1)+\E(\Kernel_{21}\mid\W_1))^2   - 4(\E\Kernel_{12})^2)\\
%     &=\frac{\I-2}{\I-1}    \V(\E(\Kernel_{12}\mid\W_1) + \E(\Kernel_{21}\mid \W_1)).
%   \end{align}
%   The ratio of these two variances tends to 1, so [[maybe cite vdvaart/serfling]]
%   \begin{align}
%     \bigg|\frac{\sqrt{\I}((\I)_2^{-1}\sum_{i\neq j}\Kernel_{ij}-\E\Kernel_{12})}{\sd(\sqrt{\I}(\I)_2^{-1}\sum_{i\neq j}\Kernel_{ij})}
%     - \frac{\sqrt{\I}(\I^{-1}\sum_{i=1}^\I(\E(\Kernel_{i0}\mid\W_i)+\E(\Kernel_{0i}\mid\W_i)-2\E\Kernel_{12})}
%     {\sd(\I^{-1/2}\sum_{i=1}^\I(\E(\Kernel_{i0}\mid\W_i)+\E(\Kernel_{0i}\mid\W_i)))}\bigg| \underset{L_2}{\to} 0.
%   \end{align}
  
%   second component:

%   The $L_2$ projection of $(\I)^{-1}_2\sum_{i\neq j}M_iN_j-\E(M)\E(N)$ onto the subspace ... is
%   \begin{align}
%     \I^{-1}\sum_{i=1}^\I(M_i\E(N)+N_i\E(M))-2\E(M)\E(N)
%   \end{align}
%   and the variance of the projection, multiplied by $\I$, is
%   \begin{align}
%     &\V(M_1\E(N)+N_1\E(M))\\
%     &=(\E N)^2(\E(M^2)-(\E M)^2) + (\E M)^2(\E(N^2)-(\E N)^2) + 2\E M\E N(\E(MN)-\E M\E N)\\
%     &= \E(M_1\E N + N_2\E M)^2 - 4(\E M\E N)^2.
%   \end{align}
%   The normalized variance of the second component of [[ref vector]] is
%   \begin{align}
%     \I\V((\I)^{-1}_2\sum_{i\neq j}M_iN_j) &=\frac{1}{\I(\I-1)^2}\V(\sum_{i\neq j}M_iN_j)\\
%                                             &=\frac{1}{\I(\I-1)^2}(O(\I^{-2}) + (\I)_3(\cov(M_1N_2,M_1N_3)+\cov(M_2N_1,M_3N_1)+2\cov(M_1N_2,M_3N_1)))\\
%     &\underset{\I\to\infty}{\to} \E(M_1\E N + N_1\E M)^2 - 4(\E M\E N)^2,
%   \end{align}
%   so the ratio of these two variances tends to $1$, and so
%   \begin{align}
%     \bigg|\frac{\sqrt{\I}((\I)_2^{-1}\sum_{i\neq j}M_iN_j - \E M\E N)}
%     {\sd(\sqrt{\I}(\I)_2^{-1}\sum_{i\neq j}M_iN_j ) }
%     - \frac{\sqrt{\I}(\I^{-1}\sum_i(M_i\E N + N_i\E M) - 2\E M\E N)}
%     {\sd(\I^{-1/2}\sum_i(M_i\E N + N_i\E M))} \bigg|
%     \underset{L_2}{\to} 0.
%   \end{align}

%   Next,
%   \begin{align}
%     &|\sqrt{\I}(\I^{-2}\sum_{i,j}M_iN_j - (\I)^{-1}_2\sum_{i\neq j}M_iN_j)|_{L_2}\\
%     &=\I\cdot\E\left( \frac{1}{(\I^2(\I-1))^2}\sum_{i\neq j,k\neq l}M_iN_jM_kN_l
%       + \frac{1}{\I^4}\sum_{i,j}M_iN_iM_jN_j - \frac{2}{\I^4(\I-1)}\sum_{i,j,k : i\neq j}M_iN_jM_kN_k \right)^2\\
%     &=\frac{\I}{(\I^2(\I-1))^2}\left(  (\I)_4(\E M\E N)^2 + O(\I^3)  \right)
%       + \frac{1}{\I^3}((\I)_2(\E(MN))^2 + O(\I))
%       -\frac{2}{\I^3(\I-1)}((\I)_3\E M\E N\E(MN) + O(\I^2)) \\
%     &\to 0.
%   \end{align}
%   This convergence implies
%       \begin{align}
%     \bigg|\frac{\sqrt{\I}(\I^{-2}\sum_{i,j}M_iN_j - \E M\E N)}
%     {\sd(\I^{-3/2}\sum_{i, j}M_iN_j ) }
%     - \frac{\sqrt{\I}((\I)_2^{-1}\sum_{i\neq j}M_iN_j - \E M\E N)}
%     {\sd(\sqrt{\I}(\I)_2^{-1}\sum_{i\neq j}M_iN_j ) } \bigg|
%     \underset{L_2}{\to} 0
%       \end{align}
%       which combined with [[first part of triangle inequality]] implies
%       \begin{align}
%     \bigg|\frac{\sqrt{\I}(\I^{-2}\sum_{i,j}M_iN_j - \E M\E N)}
%     {\sd(\I^{-3/2}\sum_{i, j}M_iN_j ) }
%     - \frac{\sqrt{\I}(\I^{-1}\sum_i(M_i\E N + N_i\E M) - 2\E M\E N)}
%     {\sd(\I^{-1/2}\sum_i(M_i\E N + N_i\E M))} \bigg|
%     \underset{L_2}{\to} 0.
%       \end{align}
  
%   Combining [[ref result for first component and result for second component]]
%   \begin{align}
%     \sqrt{\I}\begin{pmatrix}
%     \frac{(\I)_2^{-1}\sum_{i\neq j}\Kernel_{ij}-\E\Kernel_{12}}{\sd(\sqrt{\I}(\I)_2^{-1}\sum_{i\neq j}\Kernel_{ij})}\\
%     % ,        
%     \frac{\I^{-2}\sum_{i,j}M_iN_j - \E M\E N}
%     {\sd(\I^{-3/2}\sum_{i, j}M_iN_j ) }\\
%     %,
%      \frac{\I^{-1}\sum_i \Kernel_{ii}/(M_iN_i) - \E(\Kernel_{11}/M_1N_1)}
%      {\sd(\Kernel_{11}/M_1N_1)}
%   \end{pmatrix}
%     -%\\
%     \I^{-1/2}\sum_{i=1}^\I\begin{pmatrix}
%      \frac{\E(\Kernel_{i0}\mid\W_i)+\E(\Kernel_{0i}\mid\W_i)-2\E\Kernel_{12}}
%     {\sd(\E(\Kernel_{10}\mid\W_1)+\E(\Kernel_{01}\mid\W_1))}\\
%     % ,
%      \frac{M_i\E N + N_i\E M - 2\E M\E N}
%      {\sd(M_1\E N + N_1\E M)} \\
%      % ,
%      \frac{ \Kernel_{ii}/(M_iN_i) - \E(\Kernel_{11}/M_1N_1)}
%      {\sd(\Kernel_{11}/M_1N_1)}
%   \end{pmatrix}
%     =o_P(1).
%   \end{align}
% The second sequence is asymptotically normal by the CLT, so the first is as well.

\end{proof}


% \begin{tikzpicture}[scale=3]
%   \draw[step=.5cm, gray, very thin] (-1.2,-1.2) grid (1.2,1.2);
%   \filldraw[fill=green!20,draw=green!50!black] (0,0) -- (3mm,0mm) arc (0:30:3mm) -- cycle;
%   \draw[->] (-1.25,0) -- (1.25,0) coordinate (x axis);
%   \draw[->] (0,-1.25) -- (0,1.25) coordinate (y axis);
%   \draw (0,0) circle (1cm);
%   \draw[very thick,red] (30:1cm) -- node[left,fill=white] {$\sin \alpha$} (30:1cm |- x axis);
%   \draw[very thick,blue] (30:1cm |- x axis) -- node[below=2pt,fill=white] {$\cos \alpha$} (0,0);
%   \draw (0,0) -- (30:1cm);
%   \foreach \x/\xtext in {-1, -0.5/-\frac{1}{2}, 1}
%   \draw (\x cm,1pt) -- (\x cm,-1pt) node[anchor=north,fill=white] {$\xtext$};
%   \foreach \y/\ytext in {-1, -0.5/-\frac{1}{2}, 0.5/\frac{1}{2}, 1}
%   \draw (1pt,\y cm) -- (-1pt,\y cm) node[anchor=east,fill=white] {$\ytext$};
%    \end{tikzpicture}

\bibliographystyle{apalike}
\bibliography{auc.bib}


\end{document}

possible additions:
1. conditions (eg location scale) that both theta12 and theta11 are $>1/2$ or $<1/2$ simultaneously. at least can't have delta depend on M,N.
2. maybe expand literature review
3. formulas for finite sample variance with gaussian data

questions about formatting:
1. proofs seem out of order because they numbering is by order of original proof statement.
2. spacing around large conditional bar

tell lu about obu simulation--doesnt have M,N/X,Y dependence



biometrics issues
parens
author postal address
25 p limit