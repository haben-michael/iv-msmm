\documentclass{article}
\usepackage{outlines,enumitem,amsmath}
% \setenumerate[1]{label=\Roman*.}
\setenumerate[2]{label=\Alph*.}
\setenumerate[3]{label=\roman*.}
\setenumerate[4]{label=\alph*.}
\newcommand{\EE}{E}
% \newcommand{\PP}{{\mathbb{P}}}
\newcommand{\PP}{P}
% \newcommand{\E1}{\EE(Y\mid A=1,X)}
% \newcommand{\H}[1]{H^{(1)}_{#1}}
\newcommand{\E}[1]{\EE(Y\mid A=#1,X)}

\DeclareMathOperator{\Var}{Var}
\DeclareMathOperator{\Cov}{Cov}
\DeclareMathOperator{\logit}{logit}
\begin{document}

\begin{outline}[enumerate]
\textbf{Main Argument}
  \1  Background material on estimating the average treatment effect
  from the standpoint of semiparametric models, mainly summarizing
  Tsiatis 2008.
     \2 Introduce semiparametric estimator for the average treatment
  effect. Assume additional covariates are available. Introduce augmentation term, and optimized estimator that is
  semiparametric efficient given covariates.

     \2 Discuss the ``principled approach'' to estimating the regression
  terms, by estimating $E(Y\mid A=1,X)$ and $E(Y\mid A=0,X)$
  separately.

  \1 Show that the terms may be combined to $E(\tilde{Y}\mid X),$ eliminating the regressions on
  treatment. In general this requires transforming the response to
  $\tilde{Y}.$ In case the treatment propensity is $1/2$, no transformation is required.

  \1 Advantages of this simplification

     \2 Eliminate need for multiple teams/analysts etc under Tsiatis protocol.

     \2 Better estimate the conditional expectation by
     simultaneously using the data from both arms

        \3 Maybe a toy example here

           \4 improvement in standard error from estimating $E(Y\mid
           X)$ directly rather than combining estimates $\E0$ and $\E1$

           \4 improvement in point estimate in case $X$ isn't well
           balanced over the arms

  \1 Implication for studies that use residuals as response or similar
  types of standardization. In some
  fields it is common to regress out a variable of interest $Y$ on
  covariates, studying instead the residuals $Y'=Y - E(Y\mid X).$ The
  average treatment effect of $Y'$ is obtained as the slope in the
  model $E(Y') = \beta_0 + \beta_1A.$
     \2 Example from epi literature
     \2 Argue that the OLS estimate of $\beta_1$ is asymptotically equivalent to
     to the optimized estimator in case $p=P(A=1)=1/2.$
     \2 When $p\neq 1/2,$ the OLS estimate is not equivalent to the
     optimized estimator. Since the optimized estimator is
     semiparametric efficient, the OLS estimate is not.
        \3 The difference in asymptotic variances is
     \[(1-2p)^2p(1-p)\Var(\E1-\E0)\]
        which is 0 iff $p=1/2$ or the ``stratified ATEs'' $\E1-\E0$
        are constant

   \1 We can estimate both arms at the
   same time for other estimators, besides the ATE. Here the advantage is in the improved efficiency of
   estimating both rather than combining two estimates.
      \2 log-linear
      \2 logistic
      \2 discrete hazard?\\
    \end{outline}

\textbf{Simulation}
\begin{outline}[enumerate]
   \1 plot improved efficiency of using the semiparametric estimator
      over the residual OLS estimator $\hat{\beta}_1$ for a range of
      values $p\neq 1/2$ and values $\Var(\E1-\E0-\psi).$ The data is
      generated from a linear model with normal errors.
    \end{outline}

\textbf{Discussion}
    \begin{outline}[enumerate]
   \1 some drawbacks

      \2 An unscrupulous analyst may still select the model $E(Y\mid
      X)$ that optimizes the magnitude of the ATE estimate $\hat{psi}$
      if he has access to the treatment indicators. So it may still be
      necessary to have two analysts, one without access who estimates
      $E(Y\mid X)$, and one with access who then uses the first analyst's estimated CE
      to compute $\hat{psi}$.
         \3 There are a few small criticisms of Tsiatis's approach in the
         literature as well, that we might include. Eg it may be
         obvious to an analyst,  based on the response data,  who
         received treatment and who didn't.
         \2 for the other estimators (eg logistic) consistent estimators
      must be used in the weights, so the improved efficiency of
      simultaneously estimating both arms must be balanced against the
      loss of (finite-sample) efficiency of plugging in the consistent
      estimator.
    \end{outline}


\end{document}