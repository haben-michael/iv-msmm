\documentclass{article}
\usepackage{bbm}
\renewcommand{\t}[1]{\tilde{#1}}
\begin{document}

11-2

showing t-stat is asymptotically normal under alternative, aiming to
show limit under local alternatives. realized the null is when $c\to
-\infty$ at first, then thought about the nulls when the $\mu$ are
nonmonotone.

was assuming the response $y/\sigma \sim (0,1)$, which holds if the
grand mean is 0 (and the important case in practice). started thinking
about when grand mean is any $\mu$. realized that even with the simple
thresholding estimator isn't necessarily the case that thresholding
the $(\mu/sigma, 1)$ variables preserves the order of the means.


11-3

under null egger statistic is $(\mu/\sigma_j,1)$, intercept is $0$,
$\beta_1=\bar{x}/\bar{y}$. under alternative selecting by $y/\sigma >
c$, the pairs (iid?) $(y,\sigma)$ will be changed by shifting both to
the right. how does the ratio have to change to stay equal to
$\beta_1$?

focused on $\mu=0$ case. then local power is same as t-statistic,
$\mu'(0)/\sigma(c)$, where those are the post selection mean and sd.

11-5

after working on local power at $mu=0$ null, wanted to look at other
nulls--the ones found earlier, ie, the $1/\sigma$ vector plus a vector
orthogonal to the column space. got caught up trying to describe the
set of those vectors (top of p.14).

If $0<x_1<\ldots<x_n$ then there is no monotonic $v\in
C(X)^{\perp}$. $\sum v_j=0$, let
$v_1<\ldots<v_k<0<v_{k+1}<\ldots<v_n$, divide through by
$\sum_{k+1}^nv_j$ so $\sum_1^k-v_j=sum_{k+1}^nv_j=1$. then $(v,x)=0$
implies $-\sum_1^kv_jx_j=sum_{k+1}^nv_jx_j$ but the lhs is $\in
(x_1,x_k)$ whereas rhs is $in (x_{k+1},x_n)$. Same proof works if
there is some $k$ such that $v_j<0$ iff $j\le k$.

But what vectors $v$ are in fact orthogonal to the columns space? not
monontonic vectors, but maybe other permutations. ortho complement is
dimension $n-2$ so any of the components other than 2 can be ordered
arbitrarily. (11/6) Tried to show that given any $v_1,\ldots,v_n$ with
$\sum v_j=0$ and there is no $k$ with $v_1,\ldots,v_k$ all $<0$,
$v_{k+1},\ldots,v_n$ all $>0$ its orthogonal complement contains some
$0<x_1<\ldots<x_n$. But counterexample: $v=(-1/2,1/2,-1/2,1/2)$, let
$x=(a,a+b,a+b+c,a+b+c+d)$, then
$(v,x)=-(2a+b+c)/2 + (2a+2b+c+d)/2=(b+d)/2>0$.

Data are independent random pairs
$(y_j,\sigma_j),\sigma_j>0,E(y_j\mid\sigma_j)=\mu,
var(y_j\mid\sigma_j)=\sigma_j^2, j=1,\ldots,n$. Independence implies
the distribution of $(y_1,\ldots,y_n)\mid (\sigma_1,\ldots,\sigma_n)$
is the same as the distribution of
$(y_1\mid\sigma_1,\ldots,y_n\mid\sigma_n)$ (implied by bayes rule in
case densities exist, and if not, how to define conditional
distribution anyway?) The actual property we need. allows us to
replace the conditional test statistic of a meta-analysis data set say
$T((y_1,\sigma_1),\ldots,(y_n,\sigma_n))\mid\sigma_1,\ldots,\sigma_n$
with
$T((y_1\mid\sigma_1,\sigma_1,\ldots,(y_n\mid\sigma_n,\sigma_n)))$. (this
property might be better than independence. e.g. analysists might
choose sample sizes hence $\sigma_j$ based off of what other analysts
have chosen.)


Four testing scenarios: $mu=0,mu\neq 0$ and null/alt ie selection/no
selection. The observations are $(y_j,\sigma_j)$.

If no selection. Regressing $y_j/\sigma_j\sim (\mu/\sigma,1)$ on
$1/\sigma$, and the linear model
$y_j/\sigma = (1,1/\sigma)^T(\beta_0,\beta_1)+\epsilon$ is satisfied
with $\beta=(0,\mu)$ and $\epsilon=y_j/\sigma-\mu/\sigma$ independent
with equal variance. So test is consistent, asy normality for inference.

Selection present, $\mu=0$. preselection, the observations are
$y_j\sim (0,\sigma_j)$. If selection is on the p-value/z-stat
$y/\sigma\sim (0,1)$. If $y_j/\sigma_j\mid\sigma_j \sim f$ ie in
addition to
$E(y_j/\sigma_j\mid\sigma_j)=\mu/\sigma_j,var(y/\sigma_j\mid\sigma_j)=1$,
the entire conditional distribution is specified, ie the conditional
distributions $y_j\mid\sigma_j$ are a scale family
$f(y/\sigma)/\sigma$. (show can assume selection mechanism
$g(y_1/\sigma_1,\ldots,y_n/\sigma_n,U)$ can be reduced to a function
$g(y_j/\sigma_j,U)$ in this case since the $y_j/\sigma_j$ are
iid. then eg hard thresholding is given by..., probabilistic threshold
is given by...) Then postselection response is independent of
postselection regressor: given $u,v$ and selection mechanism $g_j$,
$E(u(g_j(y/\sigma_j))v(1/\sigma_j))=E(E(\ldots\mid\sigma_j))=E(E(u(g_j(y/\sigma_j))\mid\sigma_j)v(1/\sigma_j))=E(u(g_j(z)))E(v(1/\sigma_j))$
with $z\sim f$.
% then if preselection is iid, so is the
% postselection distr, say something like $(\mu(c),\sigma(c))$ if the
% same selection mechanism $g(y_j/\sigma_j)$ is used. Even if different
% selection mechanisms are used, $g_j(y_j/\sigma_j)$.
If all the selection mechanisms are the same say $g$, then
$E(g(y_j/\sigma_j)\mid1/\sigma_j)=E(g(z))$ is constant, and as before
conditional variance is constant. Again a wellspecified homoskedastic
linear model $y_j/\sigma_j \sim (1,1/\sigma_j)^T\beta+\epsilon_j$, now
with $\beta=(g(z),0)$. So test is consistent.

If distributions are different, possibly non-iid postselection
distribution. Can lead to inconsistent test. \#23 in egger.R. would be
nice to establish analytically, need estimate of t tails.

If selection mechanisms can vary ...

If $g(z)=0$ ...

Selection present, $\mu\neq 0$. egger regression response is then
$\sim (\mu_j/\sigma_j,1)$. selection will induce larger $\mu_j$ and
smaller $\sigma_j$. what can be said about postselection response? at
least in normal case, and simple thresholding as selection mechanism? ...

So if selection is present and $\mu=0$ (true null) then a type 2 error
on egger test may lead to a type 1 error on the meta-analysis. if
selection is present and $\mu\neq 0$ (true non-null), then a type 2
error on egger test may lead to exaggeration of the true non-null
effect, arguably less severe problem. either way, for practical
purposes roles of type 1 and type 2 error are reversed in the case of
eggers test, so it doesnt make sense to use the roles borrowed from
usual t-tests.

selection on p-value seems not to be a small study effect.

selection on unstandardized value.


11-10 Focused on the situation where there is selection that can
depend on both vector components, eg, $g_j(y_j,\sigma_j)$, but the
postselection distribution is the iid. e.g., the preselection
$(y_j,\sigma_j)$ arent just independent but iid, plus the selection
stragy is the same $g_j=g$, all $j$.  In this case (other case, not
iid, requires entire vector be considered), egger test is inconsistent
when the plim of betahat is 0, which is
$E(yx^2)/E(yx)=E(x^2)/E(x)$. Usual null case (not a caes of
inconsistency) is when $y=\mu+\epsilon$ is independent of
$x=1/\sigma$. Thought about local power at these distributions, but
need a parameterization. Maybe find a corresponding criterion for
inconsistency for begg test. (11-13) Rewrite criterion for
$\hat{\beta}_0=0$ as $cor(x,z)=E(z)/E(x)$ with $z\sim (0,1)$ and
$var(x)=1$ (Egger \#14). When $x$ and $z$ are independent, get $0$ on
both sides, this is just the null case. At other extreme, when the
outcome mean is the identity, $E(z\mid x)=x$, get $1$ on both
sides. (simulation at 24a.) In the normal-normal model, requires the
cutoffs to be the inverse of the gaussian hazard function
$\phi/(1-\Phi)$ (inverse exists, function is convex increasing),
analysts would have to be more selective when the studies are
larger/have smaller sd. Concave increasing selection
function. (https://math.stackexchange.com/questions/1038173/a-property-of-the-hazard-function-of-the-normal-distribution).

Not too crazy to have selection increasing in $s=1/\sigma$, if using
selection on raw value rather than p-value. Flat selection on raw
value $y_j>c$ ie $z_j\sigma_j>c$ with $z_j=y_j/\sigma_j\sim (0,1)$ is
selection increasing in $s$ viewed as selection on the p-vlue,
$z_j>c/\sigma_j$.

(11/22) Setting where given iid $(y_j,\sigma_j)$ and selecting on
$y_j=\sigma_jz_j$. Test is asymptotically null iff
$E(sz)/E(z)=E(s^2)/E(s)$ where now $s$ and $z$ are the post-selection
distributions. this criterion is obtained by taking plims in
$0=\hat{\beta_0}=\overline{y}-\overline{x}\hat{\beta_1}$. Provided
$E(z)\neq 0$ (ie the true null case) rewrite as $Es(s/E(s)-z/E(z))=0$,
where the RV in parens (say $u(s)$, after conditioning inside on $s$)
is mean zero. If $u(s)$ is monotonic in $s$, then this is not possible
unless $s$ is an atom (egger p14). In particular, this cannot happen
if $mu(cs)=E(z;z>cs)$ is the gaussian hazard function ie when the
$y_j$ are $N(0,\sigma_j)$. Try to extend to other distributions maybe
log concave distributions. (11/25) Sufficient condition is that the
integral of the surival function $\int_x^{\infty}(1-F(y))dy$ of the
studies be log concave (egger p15). This includes many commonly used
distributions. A commonly used distribution it does not include is the
pareto (egger.R \#26; bagnoli notes). (This was surprising, I thought the class
would be something like, unimodal distributions. If you move $x$ to
the rigth by $\delta$, I expected for any such distribution that
$E(Y;Y>x)$ would change by less than $\delta$.) Try to find a partial
converse; if the integral of the survival function isn't log concave,
then there is some choice of $1/\sigma$ distribution for which egger's
test is inconsistent. (12/2) Could  not establish partial converse. Just say:  $\mu'<1$ iff log concave right integral. If condition doesnt hold, will depend on joint distribution, give power law example. 

(12/2) a. Begg test is inconsistent for iid $(y,\sigma)$ iff
$E(\t{s}(\t{y}-E(\t{y})))=0$. The analogous criterion for the egger
test is $E(\t{s}(\t{y}-E(\t{y})/E(\t{s})\t{s}))$ [but before this was
$z$ not $s$--check].  b. The density of $\tilde{y}$ given
$\t{\sigma}=s$ is the same as the density of $y$ given $\sigma=s$
given the selection event (egger 16). c. The begg test is consistent
when thresholding on the raw value ie $\{y>c\}$ or p-value
$\{y/\sigma>c\}$ (egger 16), ie the criterion in a) is never met.

Possible directions to generalize. a. Outcomes models outside a scale
family, ie, given iid $(y,\sigma)$, keeping the basic meta-analysis
assumption $Var(y\mid\sigma)=\sigma^2$ but not assuming
$f(y\mid\sigma)=f_0(y/\sigma)/\sigma$. [Is this most general thing
possible, given the $\sigma$ will follow some distribution and the
pairs $(y,\sigma)$ are assumed independent plus the basic variance
assumption? No, $\sigma$ need not be identically distributed. Not sure
how much of whta was done holds just assuming $\sigma$ independent.]
b. Different selection models other than the hard thresholding on the
raw value or p-value.

(12/5) 1. The sufficienct for inconsistency of egger's test with
thresholding on the raw value is, for some $s$,
$0=u'(s)=\frac{d}{ds}\frac{(1-F_Z(cs))^\alpha}{\int_{cs}^{\infty}(1-F_Z(z))dz}$
where $\alpha=cE(S)/E(\mu_c(s))$ is $\le 1$. The monotonicity condition is probably
sufficient as well, in the sense that there is a nondegenerate distribution
of $s=1/\sigma$, such that $E(Su(S))=0$ if $u'(s)=0$ for some $s$.

2. From the monotonicity condition $u'(s)>0$ or $u'(s)<0$, gronwall's
condition gives necessary conditions that must be satisfied in order
that egger's test be consistent for raw thresholding.

3. log-concavity of the tail integral of $1-F(x)$ is implied by log
concavity of $1-F(x)$, in turn implied by log concavity of the density
$f$ (Bagnoli Thrm 2...tried to find reference in marshall olkin or
elsewhere).

4. For distributions of $S$ with a mean given by $c,m$, $u'(s)=0$ on
an interval when the outcome distribution follows a power law with
exponent $m$ and thresholding on the raw value at $c$. This seems to
be the only type of distribution where $u'(s)$ vanishes on the
interval, by solving the diff eq. (egger 16). But the criterion for
inconsistency can be met without $u$ vanishing on an interval.

5. TODO: compare local power of begg and egger at the true null. look at super-linear selection. would be nice to tie together result on tails with result on selection rate.

(12/12)
1. term ``small study bias'' may be ambiguous. No study size bias in selecting according to the unstandardized effect, $\{y>c\}$, is a small study bias when viewed as selection wrt p-value, $\{z>c\sigma\}$. No bias in selecting according to p-value, $\{z>c\}$, is a bias against larger studies when viewed as selection on the raw value, $\{y>c/\sigma\}$, the opposite of a small study effect.
2. A few days ago found what I thought was the astymptotic power function in the experiments selecting on the p-value. But it is normal and rmrks in van der vaart suggest it should be an exponential limit. Also could not establish the family is qmd. Considering now families that don't have the supports shifting wit the parameter.
3. Begg test well motivated by power. Whether selecting on raw value or p-value, clear trend of $y$ against $\sigma$.

(12/30)
Verified by simulation (egger \#34) the slope of begg test is $\mu'(0)/\sigma(0)=\int_\infty^\infty f_Z^2(z)dz * E(s_1-s2; s1<s2) / (\sqrt{4/9}$. Local power function approximation is not too bad for uniform and normal $f_Z$. (1/1) can rewrite $E(s_1-s2; s1<s2)$ as $(1/2)*E(|s_1-s_2|)$; maybe relate to mean absolute deviation?

(1/1) It is perhaps to be expected that the power to detect a trend
depends on the dispersion of $\sigma$ relative to the dispersion of
$y$. Oddly egger test/p-value thresholding power depends on location
of $\sigma$ distribution. The power curve will be better or worse than
begg's depending on this location. Is it also odd that the power
of begg's test depends on the dispersion of $\sigma$ (not relative to that of $z$

(1/7) test slope for egger test under raw thresholding see egger p 20

\end{document}