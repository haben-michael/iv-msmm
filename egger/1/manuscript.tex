\documentclass{article}
\usepackage{amsthm}
\usepackage{bbm}
\usepackage{amssymb}
\usepackage{mathtools}
\renewcommand{\t}[1]{\tilde{#1}}
\newcommand{\partiall}[1]{\frac{\partial}{\partial #1}}
\newcommand{\gm}{\theta}
\newcommand{\E}{E}
\renewcommand{\P}{P}
\newcommand{\V}{V}
\newcommand{\Cov}{Cov}
\newcommand{\Corr}{Corr}
\newcommand{\mean}[1]{\overline{#1}}
\newcommand{\sel}[1]{#1^*}
\DeclareMathOperator{\supp}{supp}
\mathtoolsset{showonlyrefs=true}

\begin{document}

\textbf{Title: Power analysis of common selection bias tests}

\begin{enumerate}

\item \emph{background}
  \begin{enumerate}

  \item \emph{background on meta-analysis, basic model}
    Meta-analysis is a ...[[copy from biometrics paper]]

    Pairs usually assumed independent, though need not be iid. key
    assumption is a common mean, the ``grand mean'' $\theta$. 
    \begin{align}
      (y_1,\sigma_1),\ldots,(y_n,\sigma_n)\\
      \E(y_j)=\theta\in\mathbbm{R}, V(y_j)=\sigma_j^2,\hspace{10px}j=1,\ldots,n
    \end{align}
    [maybe mention fixed vs random effects, we are assuming adjustment
    by between-study variance has been made and $\sigma_j^2$
    represents the unconditional variance of study $y_j$.]

  \item\emph{issue of selection, publication bias, funnel plots, formal tests: egger and begg} [maybe clarification of small sample bias defn should be here]

    [introduce two tests here]

    mention selection mechanisms here, selection on standardized statistic, on raw statistic, randomness added in.

    natural way to look for an association is a correlation between the test statsitics and $\sigma_j$, and begg's test is of this type. [define]. Egger's test a little different...

    selection on the statistic of interest can induce selection on the sigmas also.

    for selection on $y$ such as $zs>c$ one expects both $z$ and $s$ to be stochastically larger postselection and negatively dependent.

    \emph{definitions/selection on p-value seems not to be a small study effect.}
  1. term ``small study bias'' may be ambiguous. No study size bias in selecting according to the unstandardized effect, $\{y>c\}$, is a small study bias when viewed as selection wrt p-value, $\{z>c\sigma\}$. No bias in selecting according to p-value, $\{z>c\}$, is a bias against larger studies when viewed as selection on the raw value, $\{y>c/\sigma\}$, the opposite of a small study effect.

\item \emph{justification for power analysis} If selection is present
  and $\mu=0$ (true null) then a type 2 error on the selection test
  may lead to a type 1 error on the meta-analysis. If selection is
  present and $\mu\neq 0$ (true non-null), then a type 2 error on the
  selection test may lead to exaggeration of the true non-null
  meta-analysis effect. Either way, for practical purposes roles of
  type 1 and type 2 error of the selection test are reversed. The
  conservative analyst would want to bound the probability of a type 2
  error of the selection test, though there is little hope of doing so
  with such a complex alternative as ``selection bias''.  [sense to
  use the roles borrowed from usual t-tests.]  [similar issue
  encountered when using other tests for screening, e.g., meeting a
  normality assumption using a K-S test with null of normality.]
  \end{enumerate}

\item If $0<x_1<\ldots<x_n$ then there is no monotonic $v\in
  C(X)^{\perp}$. $\sum v_j=0$, let
  $v_1<\ldots<v_k<0<v_{k+1}<\ldots<v_n$, divide through by
  $\sum_{k+1}^nv_j$ so $\sum_1^k-v_j=sum_{k+1}^nv_j=1$. then $(v,x)=0$
  implies $-\sum_1^kv_jx_j=sum_{k+1}^nv_jx_j$ but the lhs is $\in
  (x_1,x_k)$ whereas rhs is $in (x_{k+1},x_n)$. Same proof works if
  there is some $k$ such that $v_j<0$ iff $j\le k$.

  But what vectors $v$ are in fact orthogonal to the columns space? not
  monontonic vectors, but maybe other permutations. ortho complement is
  dimension $n-2$ so any of the components other than 2 can be ordered
  arbitrarily. (11/6) Tried to show that given any $v_1,\ldots,v_n$ with
  $\sum v_j=0$ and there is no $k$ with $v_1,\ldots,v_k$ all $<0$,
  $v_{k+1},\ldots,v_n$ all $>0$ its orthogonal complement contains some
  $0<x_1<\ldots<x_n$. But counterexample: $v=(-1/2,1/2,-1/2,1/2)$, let
  $x=(a,a+b,a+b+c,a+b+c+d)$, then
  $(v,x)=-(2a+b+c)/2 + (2a+2b+c+d)/2=(b+d)/2>0$.



\item \emph{model} 
  Data are independent random pairs
  $(y_j,\sigma_j),\sigma_j>0,E(y_j\mid\sigma_j)=\mu,
  var(y_j\mid\sigma_j)=\sigma_j^2, j=1,\ldots,n$. Independence implies
  the distribution of $(y_1,\ldots,y_n)\mid (\sigma_1,\ldots,\sigma_n)$
  is the same as the distribution of
  $(y_1\mid\sigma_1,\ldots,y_n\mid\sigma_n)$ (implied by bayes rule in
  case densities exist, and if not, how to define conditional
  distribution anyway?) The actual property we need. allows us to
  replace the conditional test statistic of a meta-analysis data set say
  $T((y_1,\sigma_1),\ldots,(y_n,\sigma_n))\mid\sigma_1,\ldots,\sigma_n$
  with
  $T((y_1\mid\sigma_1,\sigma_1,\ldots,(y_n\mid\sigma_n,\sigma_n)))$. (this
  property might be better than independence. e.g. analysists might
  choose sample sizes hence $\sigma_j$ based off of what other analysts
  have chosen.)

  Selection on the raw value of the type $\{Z>cS\}$ affects the
  distribution of the precisions $S$ as well. The class of
  postselection distributions $\sel S$ is in principle only limited by
  the support of $Z$.
  Given $c\in\mathbbm{R}$, mean-zero $Z\sim f_Z$ and a density on
  $f$ on the nonnegative reals, such that $F_Z(cs)<1$ for any $s$ in
  the support of $f$. Then $f$ is a distribution for $\sel S$ given
  $\{Z>cS\}$ under the scale model [ref], when the unconditional distributional of $S$ is given by $g(s)=f(s)/(1-F_Z(cs))$. 
  \begin{proof}
    Since in that case,
    \begin{align}
      f_{\sel Z, \sel S}(a,b) &\propto f_{Z,S}(a,b)\{a>cb\}\\
                              &=f_Z(a)f_S(b)\{a>cb\}\\
                              &=f_Z(a)\{a>cb\}\frac{f(b)}{1-F_Z(cb)}
    \end{align}
    and integrating out $a$ gives $f_{\sel S}(b)=f(s)$.
  \end{proof}

  [may combine these small facts inone place. most are related to scale family]

  The conditional density of $\sel Y$ given $\sel S$ given a selection event $A\in\sigma(Y,S)$ is related to the conditional density of $Y$ given $S$ as
  \begin{align}
    f_{\sel Y\mid \sel S=b}(a) \propto f_{Y,S}(a,b)\mathbbm{1}_A(a,b)/\P(A) &\propto f_{Y\mid S=b}(a)\mathbbm{1}_A(a,b)/\P(A) \text{  i.e.,}\\
        \sel Y\mid (\sel S=\sel s) &\sim Y\mid (S=\sel s) \bigg| A
  \end{align}
  Ie, the two types of conditioning commute, as one can take select on
  $A$ then take the conditional density of $Y$ given $S$, or one can
  take the conditional density and then select on $A$. For example, if
  the observed data $(\sel Z,\sel S)$ comes from selection on the raw value, $\{Z>cS\}$, then one can compute the density of $\sel Z$ as $Z\mid Z>c\sel S$.
  
\item \emph{test intuition} Four testing scenarios: $\gm=0,\gm\neq 0$ and null/alt ie selection/no
  selection. The observations are $(y_j,\sigma_j)$, not necessarily iid [cite above display].

  \begin{enumerate}
  \item \emph{null/no selection} If no selection, i.e., null, and $y\sim (\gm,\sigma_j)$. Egger's
  test is the regression $y_j/\sigma_j\sim (\gm/\sigma_j,1)$ on
  $1/\sigma$. The linear model
  $y_j/\sigma = (1,1/\sigma)^T(\beta_0,\beta_1)+\epsilon$ is satisfied
  with $\beta=(0,\gm)$ and $\epsilon_j=y_j/\sigma_j-\gm/\sigma_j$
  independent with equal variance $1$. According to usual random
  design OLS theory, the test is consistent under the null, asymptotic
  normality valid for inference.[given ols assumptions eg variance
  must exist][do OLS assumptions hold if (x,y) arent necessarily iid,
  just meet the moment conditions? check wooldridge. consistency proof
  prob goes through but maybe not asy normality.]

  Begg's test. % primary study statistics are
  % $(y_j-\hat{\gm}_F)/\hat{\sigma}_j \sim (0,1) + o_p(1)$. Mean
  % is asymptotically constant with respect to the $\sigma_j$, any
  % consistent correlation tests should not pick up any
  % trend.[formalize. tried using u-stat kernel is mean 0 but actually
  % its mean is $o_p(1)$ and dont know if it goes away fast enough.]
  Must first establish that Begg's actual standardized residuals,
  $(y-\hat{\mu})/\hat{\sigma}$, is asymptotically equivalent to
  $(y-\mu)/\sigma$, ie $\sqrt{n}$ times the difference tends to
  $0$. Tried on reverse of p.13, couldnt finish. Once established, can
  refer to U-statistics asymptotic normality.

  Next, must likely use scale family assumption. Under this
  assumption, then under the null $\mu=0$, so the mean of the kernel
  of the U-statistic is just
  $\P(\{y_1/\sigma_1<y_2/\sigma_2\}\{\sigma_1<\sigma_2\})=\P(\{z_1<z_2\}\{\sigma_1<\sigma_2\})$. Under
  scale family assumption, $z_j$ indepdendent of $\sigma_j$ so the
  above is $1/4$, as required for consistency.

  Without the scale family assumption, $z_1$ and $z_2$ though both
  mean zero may have different distributions. Conssitency requires the
  median of $z_1-z_2$ be $0$, which does not [?] follow from the mean
  being zero and independence.



  
 \item \emph{alternative/Selection present}, $\gm=0$. preselection, the observations are
  $y_j\sim (0,\sigma_j)$. If selection is on the p-value/z-stat
  $y/\sigma\sim (0,1)$. If $y_j/\sigma_j\mid\sigma_j \sim f$ ie in
  addition to
  $\E(y_j/\sigma_j\mid\sigma_j)=\gm/\sigma_j,\V(y/\sigma_j\mid\sigma_j)=1$,
  the entire conditional distribution is specified, ie the conditional
  distributions $y_j\mid\sigma_j$ are a scale family
  $f(y/\sigma)/\sigma$. (show can assume any reasonable selection mechanism (ie symmetric in arguments)
  $g(y_1/\sigma_1,\ldots,y_n/\sigma_n,U)$ can be reduced to a function
  $g(y_j/\sigma_j,U)$ in this case since the $y_j/\sigma_j$ are
  iid. then eg hard thresholding is given by..., probabilistic threshold
  is given by...) Then postselection response is independent of
  postselection regressor: given $u,v$ and selection mechanism $g_j$,
  $E(u(g_j(y/\sigma_j))v(1/\sigma_j))=E(E(\ldots\mid\sigma_j))=E(E(u(g_j(y/\sigma_j))\mid\sigma_j)v(1/\sigma_j))=E(u(g_j(z)))E(v(1/\sigma_j))$
  with $z\sim f$.
  % then if preselection is iid, so is the
  % postselection distr, say something like $(\gm(c),\sigma(c))$ if the
  % same selection mechanism $g(y_j/\sigma_j)$ is used. Even if different
  % selection mechanisms are used, $g_j(y_j/\sigma_j)$.
  If all the selection mechanisms are the same say $g$, then
  $E(g(y_j/\sigma_j)\mid1/\sigma_j)=E(g(z))$ is constant, and as before
  conditional variance is constant. Again a wellspecified homoskedastic
  linear model $y_j/\sigma_j \sim (1,1/\sigma_j)^T\beta+\epsilon_j$, now
  with $\beta=(\E(g(z)),0)$. So test is consistent. [provided $\E(g(z))\neq 0$].

  If distributions are different, possibly non-iid postselection
  distribution. Can lead to inconsistent test. \#23 in egger.R. would be
  nice to establish analytically, need estimate of t tails.

  If selection mechanisms can vary ...

  If $g(z)=0$ ...

  analogous analysis for begg?
  
  Selection present, $\mu\neq 0$. egger regression response is then
  $\sim (\mu_j/\sigma_j,1)$. selection will induce larger $\mu_j$ and
  smaller $\sigma_j$. what can be said about postselection response?
  at least in normal case, and simple thresholding as selection
  mechanism? will set this case aside.
\end{enumerate}



\item\emph{criterion for consistency of egger's test} [this is the
  scale family situation, assumption needs to be introduced] Focused
  on the situation where there is selection that can depend on both
  vector components, eg, $g_j(y_j,\sigma_j)$, but the postselection
  distribution is the iid. e.g., the preselection $(y_j,\sigma_j)$
  arent just independent but iid, plus the selection stragy is the
  same $g_j=g$, all $j$.  In this case (other case, not iid, requires
  entire vector be considered), Egger's test is inconsistent when the
  plim of
  $\hat{\beta}_0=\mean{ys}-\mean{s}\hat{\Cov}(ys,s)/\hat{\V}(s)$ is
  $0$:
  \begin{align}
    % \sel\E(ys^2)/\sel E(ys)=\sel E(s^2) / \sel E(s).
    \sel\E(ys)/\sel\E(s) = \Cov(ys,s)/\V(s)
  \end{align}
  Letting $s'=s/\sqrt{\sel\V(s)}$ so that $\sel\V(s')=\sel\V(ys)=1$, rewrite as
  $\sel\Corr(ys,s')=\sel\E(ys)/E(s')$.  When $s'$ and $ys$ are independent, get
  $0$ on both sides, this is just the null case. At other extreme,
  when the conditional outcome mean is the identity, $\sel\E(ys\mid s')=s'$, get $1$ on
  both sides. [simulation at 24a.]

  Suppose the responses are normal [ref] and selection is hard
  thresholding on the standardized statistic, where the threshold may
  depend on $s$.  The conditional outcome mean
  $E(ys\mid s)=\int_{c(s)}^\infty z\phi(z)dz/(1-\Phi(c(s)))$ is the
  gaussian hazard function $\phi(c(s))/(1-\Phi(c(s)))$. this being the
  identity requires the cutoff function $c(s)$ to be the inverse of
  the gaussian hazard function $\phi/(1-\Phi)$ [inverse exists,
  function is convex increasing], a concave increasing selection
  function. [describe growth rate, should be fast. unlikely this type
  of selection] The authors of the primary studies would have to be
  more selective when the studies are larger/have smaller sd.

  Not too crazy to have selection increasing in $s=1/\sigma$, if using
  selection on raw value rather than p-value. Flat selection on raw
  value $y_j>c$ ie $z_j\sigma_j>c$ with $z_j=y_j/\sigma_j\sim (0,1)$ is
  selection increasing in $s$ viewed as selection on the p-vlue,
  $z_j>c/\sigma_j$.

  \emph{selection on raw value} Setting where given iid $(y_j,\sigma_j)$ and selecting on
  $y_j=\sigma_jz_j$. Test is asymptotically null iff
  $E(sz)/E(z)=E(s^2)/E(s)$ where now $s$ and $z$ are the post-selection
  distributions. this criterion is obtained by taking plims in
  $0=\hat{\beta_0}=\overline{y}-\overline{x}\hat{\beta_1}$. Provided
  $E(z)\neq 0$ (ie provided not in the true null case) rewrite as
  \begin{align}
    0=\E\left( s(s / \E(s) - z/\E(z))\right)=\E\left( s(s/\E(s) - \mu(s)/\E(\mu(s)))\right)
  \end{align}
  where $\mu(s)=E(z\mid s)$ is the conditional mean of $z=ys$ given $s$.
  The RV in parens, say $u(s)$ is mean zero. If $u(s)$ is monotonic in $s$, then this is not possible
  unless $u(s)$ is constant.
  \begin{proof}
    Given: $S>0$ a.s., $s_2\ge s_1$ implies $u(s_2)\ge u(s_1)$, and $E(u(S))=0$. Then $\E(u(S); u(S)<0) = -\E(u(X); u(S)>0)$, and
    \begin{align}
      \E(Su(S)) &= \E(Su(S); u(S)<0) + \E(Su(S); u(S)\ge 0)\\
                &\ge \E(Su(S); u(S)<0) + \sup\{s: u(s)<0\}\E(u(S);u(S)\ge 0)\\
                &= \E\left((S-\sup\{s: u(s)<0\})u(S); u(S)<0\}\right) \ge 0,
    \end{align}
    and $=0$ iff $S\{u(S)>0\}=\sup\{s:u(s)<0\}=S-\sup\{s:u(s)<0\}$ iff $S$ is consant or $u(S)=0$ a.s.
    % Suppose $s_2\ge s_1$ implies $u(s_2)\ge u(s_1)$. Since $E(u(s))=0$, there is some $s_0$ in the support of $s$ such that $u(s)\le 0$ for $s\le s_0$, $u(s)\ge 0$ for $s\ge s_0$, and $\int_0^{s_0}u(s)dF(s)=-\int_{s_0}^\infty u(s)dF(s)$. Then
    % \begin{align}
    %   \E\left(s\left(\frac{\mu(s)}{\E(\mu(s))}-\frac{s}{\E(s)}\right)\right)dF(s)&=\int_0^\infty s u(s) dF(s)\\
    %                                                                              &=\int_0^{s_0}+\int_{s_0}^\infty s u(s) dF(s)\\
    %                                                                              &\ge \int_{s_0}^\infty s u(s) dF(s)\ge 0.
    % \end{align}
    % The inequality is strict unless
    % $0=\int_0^{s_0}u(s)dF(s)=-\int_{s_0}^\infty u(s)dF(s)$, i.e.,
    % $u(s)$ is $0$ a.s. [[shorter proof on back of egger p15]]
  \end{proof}
  The RV $u(s)$ is degenerate iff the conditional mean $\mu(s)$ is a
  linear function of $s$. Very aggressive thresholding, like
  $c(s)$ described above.% This is In particular, this cannot happen
  % if $mu(cs)=E(z;z>cs)$ is the gaussian hazard function ie when the
  % $y_j$ are $N(0,\sigma_j)$.

  Turn therefore to when $u(s)$ is monotonic. Sufficient condition is that the
  integral of the surival function $\int_x^{\infty}(1-F(y))dy$ of the
  studies be log concave (egger p15).
  \begin{proof}
    Given $\mu(s)=\int_{cs}^\infty zf_Z(z)dz / (1-F_Z(cs))$. Then $\mu(s)>cs$ [why strict], $\E\mu(s)>c \E s$. Then
    \begin{align}
      \partiall s \left(\frac{s}{\E s} - \frac{\mu(s)}{\E \mu(s)}\right) &= \frac{1}{\E s} - \frac{\mu'(s)}{\E \mu (s)}\\
                                                                         &> (\E s)^{-1}\left( 1 - c^{-1}\mu'(s)\right).
    \end{align}
    Substitute
    \begin{align}
      \mu'(s) &= c\frac{-cs f_Z(cs)}{1-F_Z(cs)} + \frac{\left(\int_{cs}zf_Z(z)dz \right)\left(cf_Z(cs)\right)}{(1-F_Z(cs))^2}\\
      &= \frac{c f_Z(cs)}{1-F_Z(cs)}(\mu(s)-cs)
    \end{align}
    to get
    \begin{align}
      \frac{1}{\E s}\left( 1 - \frac{f_Z(cs)}{1-F_Z(cs)}(\mu(s)-cs)\right).
    \end{align}
    Let $x=cs$, then enough to show
    \begin{align}
      f_Z(x)\left(\frac{\int_x^{\infty}zf_Z(z)dz}{1-F_Z(x)}-x\right) < 1-F(x).
    \end{align}
    Integrating by parts,
    $\int_x^{\infty}zf_Z(z)dz - x +
    xF_Z(x)=\int_x^{\infty}(1-F_Z(z))dz$, provided that
    $x(1-F_Z(x))\to 0$ as $x\to\infty$. The latter is implied by
    $\V(X)=c\V(1/\sigma)<\infty$. Substituting, the condition becomes
    $\frac{\int_x(1-F_Z(z))dz}{1-F_Z(x)}<\frac{1-F_Z(x)}{f_Z(x)}$,
    $\partiall x \log \int_x(1-F_Z(z))dz < \partiall x \log
    (1-F_Z(x))$,
    $\partiall x \log \frac{1-F_Z(x)}{\int_x(1-F_Z(z))dz}>0$, implying
    since the argument to $\log$ is nonnegative
    \begin{align}
      0& < \partiall x \frac{1-F_Z(x)}{\int_x^{\infty}(1-F_Z(z))dz} \\
       & = -\frac{\partial^2}{\partial x^2}\log \int_x^{\infty}(1-F_Z(z))dz.
    \end{align}
  \end{proof}
  Log-concavity of the tail integral of $1-F(x)$ is implied by log
  concavity of $1-F(x)$, in turn implied by log concavity of the
  density $f$ [Bagnoli Thrm 2. try to find reference in marshall olkin
  or elsewhere].  This includes many commonly used distributions. A
  commonly used distribution it does not include is the pareto
  [egger.R \#26; bagnoli notes]. As for converse, proof shows $\mu'<1$
  iff log concave right integral. If $\mu'<1$ condition doesnt hold,
  monotonicity of $u$ will depend on joint distribution, give power
  law example.

  (12/5) 1. The sufficienct for inconsistency of egger's test with
  thresholding on the raw value is, for some $s$,
  $0=u'(s)=\frac{d}{ds}\frac{(1-F_Z(cs))^\alpha}{\int_{cs}^{\infty}(1-F_Z(z))dz}$
  where $\alpha=cE(S)/E(\mu_c(s))$ is $\le 1$. The monotonicity condition is probably
  sufficient as well, in the sense that there is a nondegenerate distribution
  of $s=1/\sigma$, such that $E(Su(S))=0$ if $u'(s)=0$ for some $s$.

  2. From the monotonicity condition $u'(s)>0$ or $u'(s)<0$, gronwall's
  condition gives necessary conditions that must be satisfied in order
  that egger's test be consistent for raw thresholding.


  4. For distributions of $S$ with a mean given by $c,m$, $u'(s)=0$ on
  an interval when the outcome distribution follows a power law with
  exponent $m$ and thresholding on the raw value at $c$. This seems to
  be the only type of distribution where $u'(s)$ vanishes on the
  interval, by solving the diff eq. (egger 16). But the criterion for
  inconsistency can be met without $u$ vanishing on an interval.



\item \emph{consistency of begg test}

  3. Begg test well motivated by power. Whether selecting on raw value
  or p-value, clear trend of $y$ against $\sigma$.


  (12/2) a. Begg test is inconsistent for iid $(y,\sigma)$ iff
  $E(\t{s}(\t{y}-E(\t{y})))=0$. The analogous criterion for the egger
  test is $E(\t{s}(\t{y}-E(\t{y})/E(\t{s})\t{s}))$ [but before this was
  $z$ not $s$--check].  b. The density of $\tilde{y}$ given
  $\t{\sigma}=s$ is the same as the density of $y$ given $\sigma=s$
  given the selection event (egger 16). c. The begg test is consistent
  when thresholding on the raw value ie $\{y>c\}$ or p-value
  $\{y/\sigma>c\}$ (egger 16), ie the criterion in a) is never met.

  Given a slope family $F_Z$ for the outcome distribution,
  \begin{align}
    y\mid\sigma \sim F_Z(y/\sigma)
  \end{align}
  Then $y/\sigma=ys$ is independent of $s$, as
  $y/\sigma \mid \sigma \sim f_Z$ whatever $\sigma$. Egger and begg
  tests sufficient statistics are similar,
  $(\sel y/\sel \sigma,\sel \sigma)=(\sel z,\sel\sigma)$.

  \emph{Consistency under p-value selection.} Given iid observations
  from a mean zero scale family $y\mid\sigma \sim F_Z(y/\sigma)$, suppose
  selection event $A$ is on the the standardized measurement
  $z=y/\sigma$ alone. Then the joint post selection density factors,
  \begin{align}
    f_{\sel z,\sel\sigma}(a,b)=\frac{f_{z,\sigma}(a,b)\mathbbm{1}\{a\in A\}}{\P(A)}=f_Z(a)\frac{\mathbbm\{a\in A\}}{\P(A)}f_{\sigma}(b)
  \end{align}
  and so $\sel z$ and $\sel s$ are independent. Moreover, integrating
  out $a$ in [above display] shows that $\sel s \sim s$, the
  distribution of $\sigma$ is unchanged by the selection event.

  Begg's test $\sqrt{9n/4}|\tau|>z_{1-\alpha}$ is consistent under the alternative, ie rejects with probability 1, if the mean of the kernel of Kendall's $\tau$, viewed as a U-statistic, ie the concordance probability [relate below to beggs actual statistic]
  \begin{align}
    \P\left(\left\{\frac{\sel y_1-\mu}{\sel\sigma_1}<\frac{\sel y_1-\mu}{\sel\sigma_1}\right\}\cap\{\sigma_1\ < \sigma_2\}\right)
  \end{align}
  tends in probability to a value other than 1/4 as $n\to\infty$. This holds in light of the preceding paragraph since
  \begin{align}
    \P\left(\left\{ \sel z_1-\mu/\sel\sigma_1 < \sel z_2-\mu/\sel\sigma_2\right\}\{\sel\sigma_1 < \sel\sigma_2\} \right)
    &=\P(\sel\sigma_1 < \sel\sigma_2)\E(F_{\sel z_1-\sel z_2}(\mu(1/\sel\sigma_2-1/\sel\sigma_1))\{\sel\sigma_1<\sel\sigma_2\})\\
        &=\P(\sigma_1 < \sigma_2)\E(F_{\sel z_1-\sel z_2}(\mu(1/\sigma_2-1/\sigma_1))\{\sigma_1<\sigma_2\}).
  \end{align}
  Since $\sigma_1,\sigma_2$ are iid, $\P(\sigma_1 < \sigma_2)=1/2$,
  since $\sel z_1,\sel z_2$ are iid, $F_{\sel z_1-\sel z_2}$ is
  symmetric about $0$, and on $\{\sigma_1<\sigma_2\}$,
  $1/\sigma_2-1/\sigma_1<0$. Then above is $<1/4$ so long as
  $\mu=E(\sel z)>0$. In particular, for p-val thresholding, where
  $ (1-F_Z(c))E(\sel z)=\int_c^\infty z f_Z(z)dz = -\left(\int_{-\infty}^c z f_Z(z)dz\right) > \int_{-\infty}^\infty z
  f_Z(z)dz=0$, assuming $c$ is in the interior of the support of $Z$.

  \emph{Consistency under raw selection.} Suppose
  selection event $A$ is a function of both $z$ and $\sigma$. Then no
  longer the case that $\sel z$ and $\sel\sigma$ are independent or
  that $\sel sigma$ is distributed as $\sigma$. Instead we only have
  \begin{align}
    f_{\sel z,\sel s}(a,b)=\frac{f_{z,s}(a,b)\{(a,b)\in A\}}{\P (A)}=\frac{f_{z}(a)f_s(b)\{(a,b)\in A\}}{\P (A)}
  \end{align}
  Suppose the postselection measurement $\sel z$ is monotonic in $\sel \sigma$...[got stuck, sim 36 suggest might not always be consitent...]
  
\item \emph{slope of egger and begg test, p-val thresholding}
  (12/30)
  Verified by simulation (egger \#34) the slope of begg test is $\mu'(0)/\sigma(0)=\int_\infty^\infty f_Z^2(z)dz * E(s_1-s2; s1<s2) / (\sqrt{4/9}$. Local power function approximation is not too bad for uniform and normal $f_Z$. (1/1) can rewrite $E(s_1-s2; s1<s2)$ as $(1/2)*E(|s_1-s_2|)$; maybe relate to mean absolute deviation?

  (1/1) It is perhaps to be expected that the power to detect a trend
  depends on the dispersion of $\sigma$ relative to the dispersion of
  $y$. Oddly egger test/p-value thresholding power depends on location
of $\sigma$ distribution. The power curve will be better or worse than
begg's depending on this location. Is it also odd that the power
of begg's test depends on the dispersion of $\sigma$ (not relative to that of $z$

(1/7) test slope for egger test under raw thresholding see egger p 20

\item\emph{slope, raw thresholding}
  (1/7) test slope for egger test under raw thresholding see egger p 20

\item\emph{discussion}
  random truncation model though. must find a parametrization though. exponential tilting.
\end{enumerate}


\emph{slope of egger test, p-val thresholding}

Parameterize the p-value selection
models $\{(Z,S) \mid Z>c: c\in \supp Z\}$ by the mean of $\sel
Z$. This is possible since the mean
\begin{align}
  \mu(c)=\int_c^\infty zf_Z(z)dz / (1-F_Z(c))
\end{align}
is a strictly monotonic function of the cutoff $c$ [need to assume density $f_Z(z)\neq 0$ for all $z\in\supp Z$ ie support is convex. throughout, perhaps clarify focus is on ``continuous scale families'' not just ``scale families''] ,
\begin{align}
  \mu'(c) &= \frac{-cf_Z(c)}{1-F_Z(c)} + \frac{\left(\int_c^{\infty}zf_Z(z)dz\right)f_Z(c)}{(1-F_Z(c))^2}\\
          % &= \frac{f_Z(c)}{1-F_Z(c)}\left(\frac{\int_c^\infty zf_Z(z)dz}{1-F_Z(c)}-c\right)\\
          &= \frac{f_Z(c)}{(1-F_Z(c))^2}\left(\int_c^{\infty}(z-c)f_Z(z)dz\right) > 0.
\end{align}

Let $h>0$, let $\theta_n=h/\sqrt{n}$, let $\P_n$ denote the law of $(\sel S,\sel Z)$ conditional on $\{Z>c(\theta_n)\}$. Assume $\V(S)<\infty$, $f_Z(z)\neq 0$ for $z\in\supp Z$. Then
\begin{align}
  \lim_n \P_n\left(\frac{\hat{\beta}_0}{\sqrt{\V(\hat{\beta}_0)}} >
  t_{n-1,1-\alpha}\right)= 1-\Phi\left(z_{1-\alpha} -
  \frac{h}{\sqrt{\V(Z)\E(S^2)/\V(S)}}\right).
\end{align}
So the test slope is $h/\sqrt{\V(Z)\E(S^2)/\V(S)}$.

\begin{proof}

  [0. formula for betahat]
  Egger's test is to reject when $\hat{\beta}_0/\sqrt{\hat{\V}(\hat{\beta}_0)} > t_{n-1,1-\alpha}$.
  
  Let $X_n$ be the $n\times 2$ design matrix, ie a column of $1's$ and
  a column of the regressors $s_j$. Let $\zeta_n$ by the column vector
  of measurements $\sel z_j=\sel y_j/\sel\sigma_j$. Then
  $\hat{\beta}_0$ is the first component of
  $(X_n^tX_n)^{-1}X_n^t\zeta_n$, which computes to
  $\hat{\beta}_0=(\hat{\V}(s))^{-1}\sum_{j=1}^n(\mean{s^2}-\mean{s}s_j)z_j$,
  where $\hat{\V}(s)=n^{-1}\sum_j(s_j-\mean{s})^2$.  The variance
  estimate is
  $\hat{\V}(\hat{\beta}_0)=\mean{s^2}\hat{\V}(s)RSS/(n-2)$, where
  $RSS=||(I-X(X^tX)^{-1}X^t)\zeta_n||^2$. So the test statistic is
  \begin{align}
    \frac{\hat{\beta}_0=(\hat{\V}(s))^{-1}\sum_{j=1}^n(\mean{s^2}-\mean{s}s_j)z_j}{\mean{s^2}\hat{\V}(s)RSS/(n-2)}.
  \end{align}
  
  [1. First step: make iid: 1--variance term, 2--means]
  To show: Asymptotic equivalence of
  \[
    n^{-1/2}\frac{\sum_{j=1}^n\left(\V(S)^{-1}(\E(S^2)-\E(S)S_j)Z_j-\mu_n\right)}{\sqrt{\V(Z)\E(S^2)/\V(S)}}
    \]
  and
  \[
    n^{-1/2}\frac{\sum_{j=1}^n\left(\V(S)^{-1}(\E(S^2)-\E(S)S_j)Z_j-\mu_n\right)}{\sqrt{\V(Z)\E(S^2)/\V(S)}}.
  \]
  To show:
  \begin{align}
    \sqrt{n}\left(\left(\E(S^2)-\mean{S^2}\right)\mean{\sel{Z}} - \mean{S\sel{Z}}\left(\E(S)-\mean{S}\right)\right) \underset{P_n}{\to} 0
  \end{align}
  where convergence is in probability under the sequence of laws $\P_n$ of $(\sel Z,\sel S)$ at $\theta_n$.

  First term: $\sqrt{n}\left(\E(S^2)-\mean{S^2}\right)$ is $O_{P_n}(1)$
  by the CLT, also distribution of $S$ does not change with $P_n$. [need
  $V(S)<\infty$] and $\mean{\sel{Z}}\to 0$ using a weak LLN for
  triangular arrays and
  $\sel Z \sim_{\theta_n} Z\{Z>c_n\}/(1-F_Z(c_n))$. So
  $\sqrt{n}\left(\E(S^2)-\mean{S^2}\right)\mean{\sel{Z}}\to 0$ in
  probability along $P_n$. Second term:
  $\sqrt{n}\left(\E(S)-\mean{S}\right)$ is $O_{P_n}(1)$ by CLT. Orthogonality
  of $S$ and $\sel{Z}$ and domination of $\sel{Z}$ implies
  $\mean{S\sel{Z}}\to_{P_n}\E_0(SZ)=0$.

  To show:
  $RSS/n=\zeta_n^t(I-X(X^tX)^{-1}X^t)\zeta_n/n \to_{\P_n}\V(Z)$,
  convergence is in probability along $\P_n$. 
  \begin{align}
    \zeta_n^tX(X^tX)^{-1}X^t\zeta_n &= (\hat{\V}(s))^{-1}\left(\mean{\sel{z}}(\mean{s^2}\mean{\sel{z}}-\mean{s}\mean{s\sel{z}})+\mean{\sel{z}s}(\mean{\sel{z}s}-\mean{\sel{z}}\mean{s}) \right)\\
    &= (\hat{\V}(s))^{-1}\left(\mean{s^2}(\mean{\sel{z}})^2 + (\mean{\sel{z}s})^2 - 2\mean{s}\mean{\sel{z}}\mean{s\sel{z}}\right).
  \end{align}[fix overbar leaking over like unibrow]
  Converges in probability to $0$ as above [verify], with each
  monomial convering to $\E(S^2)\E(Z^2)$. So $RSS/n= o_{\P_n}(1)+ n^{-1}\zeta_n^t\zeta_n$, which tends along $\P_n$ to $\E(Z^2)=\V(Z)$, as above.
  
  [2. Second step: CLT application]

  Let $\sel Z_n$ denote the postselection distribution of $Z$ under $\theta_n=h/\sqrt{n}$, i.e., conditional on $\{Z>Sc(\theta_n)\}$ [causes confusion with indexing subscript]. Let $E_n$ denote expectation under $\theta_n=h/\sqrt{n}$, i.e., conditional on $\{Z>Sc(\theta_n)\}$. Only relevant for $\sel Z$ since the distribution of $\sel S\sim S$ does not change with $n$.  Let  $\mu_n=\E(\sel Z_n)=h/\sqrt{n}$.

  Apply Lindeberg-Feller CLT to conclude asy normality:
  \begin{align}
    n^{-1/2}\frac{\sum_{j=1}^n\left(\V(S)^{-1}(\E(S^2)-\E(S)S_j)Z_j-\mu_n\right)}{\sqrt{\V(Z)\E(S^2)/\V(S)}} \overset{\theta_n}{\Rightarrow} N(0,1)
  \end{align}
  
  Since the $(\sel Z_j,S_j)$ are iid, the Lindeberg condition is
  \begin{align}
    \E_n\left(\left( \frac{\V(S)^{-1}(\E(S^2)-\E(S)S_1)\sel{Z_1}-\E_n(\sel Z)}{\sqrt{\V(Z)\E(S^2)/\V(S)}}  \right)^2; n^{-1/2}|\ldots|>\epsilon\right) \to 0
  \end{align}
  for all $\epsilon>0$. Ellipses represent the term in parenthesis. The family $\{\V(S)^{-1}(\E(S^2)-\E(S)S_1)\sel{Z_1}-\E_n(\sel Z)\}$ over the probabilities $\P_n$ is in fact uniformly integrable. Since the distribution of $S_1$ does not depend on $\P_n$, $S_1$ and $\sel Z_1$ are independent, and $\sel Z_1 \sim_{\theta_n} Z\{Z>c_n\}/(1-F_Z(c_n)) \to_{a.s.} Z$.


  Also
  \begin{align}
    \E_n\left(\left(\frac{(\E(S^2)-\E(S)S_1)\sel Z_1}{\V(S)} -\E_n(\sel Z)  \right)^2\right) &= \frac{\E_n((\sel Z)^2)((\E(S^2))^2-\E(S^2)(\E(S))^2)}{\V(S)^2} - (\E_n(\sel Z))^2\\
                                                                                             &= \E_n((\sel Z)^2)\E(S^2)/\V(S) - (\E_n(\sel Z))^2\\
                                                                                             &\rightarrow \frac{\V(Z)}{\V(S)}\E(S^2)
  \end{align}
  as $n\to\infty$. By definition of $E_n$, $\E_n(\sel Z)\to 0$ and as
  above domination of $\sel Z$ by bounded multiples of $Z$ implies
  $\E_n(\sel Z)\to \E(Z^2)$. So the summands are standardized.

  [  3. Third step: obtain slope]
  
  The local limiting power at the null $\theta=0$ is then
  \begin{align}
    &\lim_n \P_n\left(\frac{\hat{\beta}_0}{\sqrt{\V(\hat{\beta}_0)}} > t_{n-1,1-\alpha}\right)
      = \lim_n\P_n\left(n^{-1/2}\frac{\sum_{j=1}^n\V(S)^{-1}(\E(S^2)-\E(S)S_j)Z_j}{\sqrt{\V(Z)\E(S^2)/\V(S)}} > t_{n-1,1-\alpha} \right)\\
    &= \lim_n\P_n\left(n^{-1/2}\frac{\sum_{j=1}^n\left(\V(S)^{-1}(\E(S^2)-\E(S)S_j)Z_j - \mu_n\right)}{\sqrt{\V(Z)\E(S^2)/\V(S)}} > t_{n-1,1-\alpha} -\frac{n^{-1/2}\mu_n}{\sqrt{\V(Z)\E(S^2)/\V(S)}}   \right)\\
    &= 1-\Phi\left(z_{1-\alpha} - \frac{h}{\sqrt{\V(Z)\E(S^2)/\V(S)}}\right).
  \end{align}
  
\end{proof}

\end{document}
